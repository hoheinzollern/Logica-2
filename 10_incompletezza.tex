
%%%%%%%%%%%%%%%%%%%%%%%%%%%%%% LyX specific LaTeX commands.
\newcommand{\noun}[1]{\textsc{#1}}
\providecommand*{\perispomeni}{\char126}

\chapter{Incompletezza}

\begin{abstract}
Queste note sono divise in quattro momenti: dopo un inquadramento
storico del problema dei fondamenti della matematica prenderemo in
considerazione alcuni aspetti del pensiero di David Hilbert esponendo
in particolare il suo programma fondazionale, quindi parleremo delle
conseguenze fondazionali dei teoremi di incompleteza di Godel sottolineando
le difficoltà in cui si imbattè il programma di Hilbert a seguito
di questi ed infine accenneremo allo stato dell'arte in ambito fondazionale.
\end{abstract}

\section{Introduzione}

Non è fuori luogo individuare le origini di gran parte della matematica
contemporanea nel diciannovesimo secolo, un'età di grandi trasformazioni
qualitative e un'età durante la quale la matematica ha subito cambiamenti
così profondi che non è esagerato parlare di una vera e propria seconda
nascita della materia così come nell'età greca c'era stata la prima%
\footnote{J.Avigad \& E.H.Rech, {}``Clarifyng the nature of infinite'': the
development of metamathematics and proof theory, pg.5%
}. Analisi, geometria, algebra e logica matematica vengono completamente
rivoluzionate.

L'analisi dopo un secolo di travolgenti successi incontra alcune difficoltà
che inducono a un generale ripensamento. E' la fase della cosiddetta
rigorizzazione dell\textquoteright{}analisi, nella quale i ferri del
mestiere, cioè i concetti di funzione, serie, integrale, continuità,
ecc., vengono affinati (da Cauchy, Fourier, Riemann e tanti altri)
fino a liberarli dall\textquoteright{}uso degli infinitesimi e infiniti
attuali, tramite il concetto di limite (l\textquoteright{}analisi
$\epsilon - \delta$ di Weierstrass). Un passo
essenziale dell\textquoteright{}impresa, e ultimo in ordine cronologico
(siamo già nel 1872), è la definizione rigorosa dei numeri reali (razionali
e irrazionali) ad opera di Weierstrass, Dedekind e Cantor%
\footnote{Sambin, Alla ricerca della certezza perduta, pg.7%
}.

La geometria nel tentativo di dimostrare la dipendenza del quinto
postulato della geometria euclidea dagli altri si imbatte nelle geometrie
non-euclidee. La scoperta di tali geometrie(Gauss, Lobacewskij, Bolyai)
mette in crisi la nozione di intuizione geometrica e l'idea che gli
assiomi debbano necessariamente codificare un'aspetto univoco ben
determinato della nostra concezione \emph{dello spazio}. Simultaneamente,
con la nascita e l'evolversi della geometria proiettiva, si giunge
alla definizione di un nuovo livello di indagine: l'intuizione geometrica
viene così decomposta in dati ancora più semplici, connessi con lo
studio degli {}``aspetti grafici'' dello spazio. In tale situazione
si perde la nozione di un referente oggettivo privilegiato, al quale
riportare tutte le nostre teorizzazioni. Si assiste così ad una progressiva
divericazione tra geometria fisica - intesa come studio delle proprietà
dello spazio fisico - e geometria pura%
\footnote{Cantini, I fondamenti della matematica, pg.17%
}.

L'algebra sulla scorta di interessanti progressi nelle sue teorie
classiche assume una forma completamente astratta. Tra le altre cose
notevoli sono da ricordare gli sviluppi in teoria delle equazioni
da parte di Galois, la nascita della scuola britannica di algebristi
(che include personaggi quali Hamilton, Cayley, Boole e Sylvester),
il lavoro di Grassman sulle algebre che generalizzano i quaternioni
di Hamilton. Si assiste inoltre ad un netto aumento delle interazioni
tra l'algebra e le altre aree della matematica: in geometria metodi
algebrici sostituiscono i tradizionali metodi sintetici, la stessa
cosa succede in teoria dei numeri grazie a Kummer e Dedekind, Abel
applica metodi algebrici allo studio di funzioni ellittiche e metodi
simili si sviluppano in tutta l'analisi grazie a Jacobi, Dirichlet,
Riemann e Hamilton. Brevemente, tra l'inizio e la fine del diciannovesimo
secolo l'interesse per l'algebra e il suo posto in matematica cambiano
enormemente fino al manifestarsi di una vera e propria \emph{algebrizzazione
della matematica}%
\footnote{J.Avigad \& E.H.Rech, {}``Clarifyng the nature of infinite'': the
development of metamathematics and proof theory, pg5.%
}.

La logica matematica dal canto suo subisce una rigorosa definizione
dei suoi linguaggi formali e una chiara formulazione dei concetti
che le sono propri, sia dal punto di vista sintattico che dal punto
di vista semantico. Viene infatti definito il \emph{linguaggio della
logica dei predicati del primo ordine}(Frege e Peano), di gran lunga
più espressivo dei linguaggi della logica classica che permettevano
uno studio esclusivo della sillogistica escludendo il trattamento
delle relazioni, vengono formulati i primi \emph{calcoli logici} in
cui le regole o gli assiomi che caratterizzano l'apparato deduttivo
sono caratterizzati esplicitamente, inizia lo studio algebrico della
logica(Boole e De Morgan) che permette una organica carattterizzazione
semantica delle leggi logiche ed infine viene chiarita in modo appropriato
la distinzione tra modello e teoria formale passo fondamentale verso
la nascita della \emph{teoria dei modelli}.

Questa rivoluzione nelle scienze matematiche porta ad un'acceso dibattito
sulla definizione dell'oggetto proprio degli studi matematici e quindi
a interrogativi sulla natura della disciplina stessa. La tradizionale
definizione della matematica quale scienza della grandezza, della
misura e della quantità infatti non appariva più adeguata. Questo
dibattito che coinvolge personaggi quali Boole, Kroneker, Dedekind
e Cantor segna la presenza di due possibili visioni della matematica,
l'una di natura sintattica l'altra di natura semantica. La matematica
infatti può essere vista da un lato come la scienza della rappresentazione
simbolica e del calcolo e dall'altro come la scienza del ragionamento
concettuale circa strutture caratterizzate astrattamente %
\footnote{Vedremo che la distinzione tra queste due differenti concezioni della
matematica e della logica sarà un argomento importante nel commento
ai teoremi di incompletezza di Godel.%
}.

Legata a quest'esigenza di caratterizzazione della scienza matematica
è un'esigenza di unificazione del sapere matematico ormai frazionato
in una miriade di teorie interdipendenti, questo bisogno sarà soddisfatto
da quella che è ancora oggi una delle teorie più discusse e interessanti
di tutta la matematica, la teoria degli insiemi la cui nascita è dovuta
al matematico tedesco Georg Cantor.

Cantor affrontando uno specifico problema di analisi(sulle serie di
Fourier), è indotto a misurare aggregati arbitrari di numeri reali
e ciò lo conduce a tentare un trattamento matematico del concetto
di infinito. Nel realizzare questo intento in pochi anni egli sviluppa
una teoria dei numeri ordinali e cardinali transfiniti e dagli aggregati
di numeri reali giunge alla nozione astratta di insieme.

Dopo alcuni anni di diretta opposizione o al più indifferenza i matematici
cominciano a vedere di quale nuovo paradiso Cantor avesse aperto le
porte. La potenza del linguaggio insiemistico conduce così alla nascita
di nuove teorie (la topologia. l'analisi funzionale, la geometria
algebrica, la teoria della misura di Lebesgue, ...) e al rinnovamento
di teorie già affermate (l'algebra astratta, la geometria algebrica,
l'analisi reale, ...)%
\footnote{Sambin, Alla ricerca della certezza perduta, pg.8%
}. In questo modo la teoria degli insiemi si afferma quale linguaggio
e strumento di indagine matematico, cioè come repertorio di concetti
in funzione dei quali esprimere tutti gli altri concetti matematici.

Il paradiso di Cantor però finirà in breve tempo nel rivelarsi un
inferno più che un paradiso. Infatti intorno al 1900, prima timidamente
Cantor poi gli italiani che tra i primi avevano recepito le sue idee
quindi Russell e altri, si rendono conto di alcuni paradossi insiti
nella teoria%
\footnote{Sambin, Alla ricerca della certezza perduta, pg.8%
}. 

Visto il ruolo fondazionale attribuito alla teoria degli insiemi la
comparsa dei paradossi rende pressante il bisogno di una sicura fondazione
della matematica. Così la ricerca sui fondamenti, che nell'Ottocento
era vista sostanzialmente come sistemazione rigorosa di conoscenze
e risultati non discussi, e quindi parte del progresso e compito esclusivamente
matematico, nei primi anni del Novecento inizia ad essere considerata
come mezzo necessario, per quanto poco gradito, per scongiurare una
ritirata. E' la cosiddetta crisi dei fondamenti%
\footnote{Sambin, Alla ricerca della certezza perduta, pg.8%
}.

E' un momento delicato, in cui emergono divergenze così profonde sul
modo di concepire la matematica che, forse per la prima volta, si
hanno come conseguenza diversi modi di fare matematica%
\footnote{Sambin, Alla ricerca della certezza perduta, pg.9%
}. Numerose sono le proposte fondazionali che si affacciano sulla scena.
Gli indirizzi principali sono tre: \emph{(1)Logicismo}, \emph{(2)Formalismo}
e \emph{(3)Intuizionismo}. Di fatto solo il seconso e il terzo avranno
ripercussioni sulla pratica matematica, concretizzandosi nella matematica
formalista da una parte e nella matematica costruttiva dall'altra.

David Hilbert può essere considerato come il padre dell'approccio
formalista.


\section{Hilbert}

David Hilbert nasce a Konisberg nel 1862, compie nella sua città natale
l'intero curriculum di studi e vi inizia la carriera universitaria
divenendo dapprima (1886) \emph{Privatdozent }e poi (1892) \emph{Extraordinarius.}
In quegli anni entra in contatto personale con buona parte dei matematici
più rappresentativi del suo tempo, per esempio Karl Weierstrass, Richard
Dedekind e Georg Cantor.

Gli interessi matematici di Hilbert furono vastissimi, di modo che
è difficile trovare un settore della matematica nel quale egli non
sia intervenuto con contributi di tale peso ed importanza da indurre
in essi profonde trasformazioni tematiche e metodiche%
\footnote{Abrusci, Introduzione, pg.15%
}. Per citarne alcuni, tra il 1888 e il 1893 risolve quello che era
conosciuto come il {}``problema fondamentale della teoria degli invarianti'',
sempre nel 1893 riunifica gli sviluppi in teoria dei numeri dovuti
a Gauss, Kronecker e Dedekind fornendo una stabile fondazione alla
teoria algebrica dei numeri, sviluppa una organizzazione assiomatica
delle geometrie euclidea e non-euclidea ed infine lavora sulle equazioni
integrali e su importanti problemi in fisica matematica. Hilbert inoltre
accompagnò sempre questi lavori con approfondite ricerche fondazionali
che dal 1917 in poi finirono addirittura con il divenire il centro
principale della sua attività di ricerca.

Il lavoro di Hilbert illustra gran parte dei temi e degli interessi
della matematica del diciannovesimo secolo: l'enfasi per la rappresentazione
simbolica e per la caratterizzazzione astratta, l'uso di metodi infinitari
e non costruttivi e la ricerca di un'unità fondazionale%
\footnote{J.Avigad \& E.H.Rech, {}``Clarifyng the nature of infinite'': the
development of metamathematics and proof theory, pg.16%
}. E proprio la ricerca di quest'unità e di nuovi metodi matematici
lo renderà uno degli artefici della insiemizzazione della matematica
e della conseguente rivalutazione della teoria degli insiemi sviluppata
da Cantor. 

In virtù di tutto ciò egli vide nei paradossi della teoria degli insiemi
un problema drammatico e di cui era necessaria una chiara e precisa
soluzione al fine di restituire alla matematica quella certezza e
inoppugnabilità che a suo giudizio le era propria. La sua risposta
alla crisi dei fondamenti fu la proposizione di una nuova forma di
metodo assiomatico\emph{: il metodo assiomatico formale.}


\subsection{Metodo assiomatico}

Hilbert nel suo scritto \emph{Pensiero Assiomatico }definisce una
teoria, sia essa matematica o meno, come una \emph{intelaiatura di
concetti} e nota poco dopo che quando esaminiamo più in profondità
una determinata teoria, ogni volta riconosciamo che alla base della
costruzione dell'intelaiatura dei concetti ci sono poche e ben individuate
proposizioni e che queste sole bastano a costruire da esse, secondo
principi logici, l'intera intelaiatura. Hilbert chiarisce che queste
basi sono i teoremi fondamentali della teoria e che esse stesse poggiano
su altre proposizioni che sono irriducibili ad altre, cioè non dimostrabili
in funzione di altre, queste proposizioni sono gli assiomi della teoria%
\footnote{Hilbert, Fondamenti della matematica, pg.178%
}.

Per Hilbert quindi assiomatizzare una teoria significa comprenderne
l'architettura interna e organizzarla in base ad essa ed è proprio
in virtù di questa concezione dell'assiomatica che egli propone che
ogni teoria matematica venga formalizzata e assiomatizzata. In questo
modo una teoria matematica diviene un oggetto astratto che può avere
più significati, cioè può venire interpretata in più strutture. Questo
spostamento di piano comporta però che gli assiomi di una teoria non
possano più essere giustificati in base alla loro evidenza, è necessario
cioè individuare altri criteri che ci permettano di discriminare una
buona teoria da una cattiva, questi criteri sono individuati da Hilbert
nell\emph{'indipendenza e nella non-contraddittorietà} degli assiomi
della teoria. 

Con il passaggio dall'assiomatica classica all'assiomatica formale
si assiste perciò ad uno spostamento di interesse, dalla verità di
un insieme di enunciati contenutistici alla non-contraddittorietà
di un insieme di enunciati formali. La non-contraddittorietà diviene
in questo modo ciò che era la verità per la matematica precedente
alle rivoluzioni teoriche ottocenetesche. 

Per Hilbert inoltre il metodo assiomatico oltre ad essere uno strumento
di organizzazione e di indagine è uno strumento di salvaguardia della
matematica, esso infatti a suo giudizio sarebbe in grado di esimerla
da possibili contraddizioni in modo da renderla certa ed inoppugnabile.
Ciò lo spinge a voler estendere il metodo assiomatico a tutta la scienza
in modo da rendere sicuro l'intero sapere scientifico. Con le sue
parole:

Tutto ciò che può essere oggetto del pensiero scientifico, non appena
è maturo per la formazione di una teoria, cade sotto il metodo assiomatico
e per suo tramite sotto la matematica. Progredendo verso livelli sempre
più profondi di assiomi otteniamo anche illuminazioni sempre più profonde
sulla natura del pensiero scientifico e diveniamo sempre più consapevoli
dell'unità del nostro sapere. Nel segno del metodo assiomatico la
matematica sembra chiamata ad un ruolo guida in tutto ciò che è scienza%
\footnote{Hilbert, Fondamenti della matematica pg.188%
}.

In questa fase del suo pensiero(1917-1918) Hilbert inizia a tracciare
una chiara distinzione tra nozioni semantiche e nozioni sintattiche
in logica e matematica. In uno scritto di questo periodo infatti nota
esplicitamente che i linguaggi e le teorie formali possono venire
interpretati in una struttura matematica ma che i sistemi deduttivi
possono essere studiati anche autonomamente, indipendentemente da
una interpretazione semantica%
\footnote{J.Avigad \& E.H.Rech, {}``Clarifyng the nature of infinite'': the
development of metamathematics and proof theory, pg.20%
}. 

Hilbert inizia così a rendersi conto della possibilità di uno sviluppo
esclusivamente sintattico di una teoria matematica e da questo momento
in poi come vedremo nel prossimo paragrafo prediligerà metodi proof-teoretici
nei suoi lavori fondazionali. E' bene notare però che un conto è il
programma fondazionale di Hilbert un conto è la sua concezione della
matematica, egli infatti da grande matematico quale era conosceva
bene il pensiero concettuale proprio della pratica matematica e quindi
il valore della componente semantica nelle scienze deduttive. L'intento
di Hilbert non era quello di ridurre la matematica ad una manipolazione
di segni ma di trattarla in questo modo da uno punto di vista metamatematico
al fine di dimostrane la coerenza e quindi porre a riparo da ogni
rischio il pensiero semantico, quello che per lui era la vera matematica.
Le ragioni del perseguimento da parte di Hilbert di una prospettiva
proof-teoretica in ambito fondazionale sta quindi nel tentativo di
tracciare una chiara linea di demarcazione tra questioni matematiche
e metamatematiche. Il suo errore è stato solo nel considerare l'approccio
sintattico e quello semantico come equivalenti e quindi il prospettare
la possibilità di ridurre l'uno all'altro, cioè di modellare completamente
il pensiero astratto tramite tecniche di manipolazione simbolica.


\subsection{Il programma di Hilbert}

Veniamo ora al cuore della ricerca fondazionale hilbertiana. Per Hilbert,
fondare la matematica significa dimostrarne la coerenza, cioè dimostrare
la coerenza della totalità delle teorie matematiche e con esse quella
delle teorie scientifiche che si basano su di esse. A tal fine ogni
teoria matematica va assiomatizzata secondo il metodo assiomatico
formale. 

Le dimostrazioni di coerenza delle teorie matematiche devono avvenire
tramite metodi esclusivamente \emph{proof-teoretici, }senza sfruttare
tecniche \emph{model-teoretiche }quali l'esibizione di un modello
per la teoria. Il programma ha quindi una natura puramente \emph{sintattica},
esso deve svolgersi senza alcun riferimento a considerazioni semantiche.
E' in questo senso che va letta la discussa affermazione secondo la
quale la matematica non sarebbe altro che una manipolazione di segni. 

Le tecniche \emph{proof-teoretiche }chiamate in causa si fondano su
particolari metodi matematici, i cosiddetti \emph{metodi finitisti},
i quali fanno riferimento ad una matematica ritenuta {}``contenutisticamente
sicura''' e che non necessita di fondazione. Viene in questo modo
a costituirsi una nuova disciplina matematica che costruisce i formalismi
e con mezzi contenutisticamente sicuri dimostra la coerenza delle
teorie assiomatiche formali, questa disciplina è la metamatematica
o teoria della dimostrazione.

Ora, cosa questi metodi finitisti siano non è molto chiaro ed è questione
dibattuta sin dai tempi di Hilbert. Quello che possiamo dire è che
Hilbert distingue all'interno della matematica tra una {}``matematica
finitaria'', dotata di concetti intuitivi e di verità evidenti che
usa l'infinito solo nella sua forma potenziale e procede con mezzi
concreti e finiti, e una {}``matematica transfinita''%
\footnote{Abrusci, Introduzione, pg.33%
}. Nella matematica finitaria è accettato anche il concetto intuitivo
di numero naturale e in essa è possibile svolgere parte dell'aritmetica.

In ogni caso per Hilbert gli enti simbolici di un formalismo (simboli,
successioni di simboli, successioni di successioni di simboli, ...)
vanno pensati come numeri, le proprietà e le relazioni di tali enti
(quali {}``essere formula'', {}``essere una variabile'', {}``essere
una dimostrazione'', ...) come proprietà e relazioni di numeri e
le operazioni su tali enti sono come operazioni su numeri.

Il programma fondazionale hilbertiano si distingue perciò oltre che
per la sua natura \emph{sintattica} di cui si è parlato sopra per
la sua natura \emph{autofondativa}, Hilbert intende ciò fondare la
matematica. intesa come famiglia di teorie assiomatiche formali, con
la matematica(metodi finitisti). 

Vedremo che il primo ed il secondo teorema di Godel ridimensionano
le possibilità di realizzazione di un tale programma, facendo riferimento
proprio ai limiti di un approccio esclusivamente sintattico e all'impossibilità
di una prospettiva autofondativa.


\section{Incompletezza}

Il concetto di completezza e il suo complementare (incompletezza)
sono concetti molto importanti in logica formale. Sarà quindi bene
fare alcune precisazioni a riguardo.

Ci sono diversi tipi di completezza, in particolare è bene distinguere
tra completezza di una logica e completezza di una teoria formale.

\emph{\noun{Definizione 1. }}Una logica si dice completa se esiste
un calcolo logico capace di dedurre tutte le verità logiche della
data logica.

Per quanto riguarda le teorie ci sono poi due tipi di completezza,
una sintattica e una semantica.

\emph{\noun{Definizione 2}}. Una teoria si dice sintatticamente completa
se per ogni formula $\varphi$ del linguaggio della teoria o $\vdash\varphi$
o $\vdash\neg\varphi$.

\emph{\noun{Definizione 3. U}}na teoria si dice semanticamente completa
se per ogni formula $\varphi$ del linguaggio della teoria o $\models\varphi$
o $\models\neg\varphi$.

Aggiungiamo a queste tre definizioni una quarta, la definizione di
categoricità, in quanto questa proprietà delle teorie formali può
essere considerata come una forma di completezza.

\emph{\noun{Definizione 4. }}Una teoria si dice categorica se essa
ammette un unico modello a meno di isomorfismi.

La completezza di una logica assieme alla sua correttezza stabilisce
di fatto un'equivalenza tra un approccio sintattico e uno semantico
nei confronti della logica in esame. Non tutte le logiche sono complete,
in particolare la logica proposizionale e la logica del primo ordine
sono complete%
\footnote{La completezza della logica del primo ordine è stata dimostrata per
la prima volta proprio dallo stesso Godel nel 19..%
}, mentre la logica del secondo ordine non lo è%
\footnote{L'incompletezza di questa logica, teorema di Church, è un corollario
del secondo teorema di incompletezza di Godel.%
}.

La completezza sintattica, anche detta completezza deduttiva, stabilisce
la possibilità di uno sviluppo puramente sintattico di una teoria.
La completezza semantica stabilisce invece la possibilità di caratterizzazione
formale di una particolare struttura. Non sempre le due completezze
sono distinte, se infatti l'apparato deduttivo della teoria è completo
i concetti di deducibilità e verità in tutti i modelli sono equivalenti
e quindi le due completezze coimplicantesi, in assenza però di tale
condizione può darsi il caso che se ne verifichi una senza che si
verifichi l'altra.

La categoricità è una proprietà di natura puramente semantica, essa
può essere considerata come una forma di completezza in quanto assicura
una sorta di completezza descrittiva della struttura che è modello
della teoria. Questa proprietà è molto forte tanto da implicare la
completezza semantica e via completezza dell'apparato deduttivo della
teoria in esame la completezza sintattica.

Il primo teorema di incompletezza di Godel fa riferimento alla completezza
deduttiva mentre il secondo teorema di incompletezza non rientra in
nessuno dei casi precedenti. Esso è così definito solo con abuso di
terminologia, sarebbe infatti più preciso definirlo, come fanno i
testi tecnici %
\footnote{Cfr. Boolos \& Jeffrey, Computability and logic%
}, come teorema di non-provabilità della consistenza.


\subsection{Primo teorema di incompletezza}

Il primo teorema di incompletezza di Godel afferma che se l'aritmetica
formale al primo ordine è coerente allora essa non è deduttivamente
completa, esitono cioè nella teoria delle \emph{proposizioni indecidibili
}che il sistema non è in grado nè di dimostrare nè di refutare. 

Per provare il suo teorema Godel elabora uno strumento che permette
la realizzazione effettiva della visione hilbertiana della metamatematica,
la cosiddetta\emph{ aritmetizzazione della sintassi}. Essa consiste
in una codifica delle componenti sintattiche della teoria in modo
da far corrispondere ad ogni termine, ad ogni formula e ad ogni derivazione
uno ed un solo numero naturale. Tutta la metalogica può così essere
espressa aritmeticamente e trattando la teoria di aritmetica essa
è in grado di esprimere la propria sintassi. In questo modo la metalogica
e la logica vengono ricondotte allo stesso livello.

E' proprio grazie a questo espediente che Godel è in grado di esibire
una proposizione indecidibile, essa è la cosiddetta proposizione G
che in termini informali si può considerare come asserente la propria
indimostrabilità. Nonostante tale proposizione non sia dimostrabile
nella teoria essa però è da noi riconosciuta come vera il che significa
che ci sono delle verità aritmetiche che non sono dimostrabili nell'aritmetica
formale. Tutto ciò segna un limite nelle capacità deduttive della
teoria formale e la cosa potrebbe portare a considerare tali limiti
come propri del metodo assiomatico stesso ma a ben vedere le cose
non stanno proprio così.

La teoria formale cui si riferisce il teorema di Godel è l'aritmetica
formale al primo ordine ma essa non è l'unica teoria formale per l'aritmetica,
l'aritmetica può essere formalizzata anche al secondo ordine e anzi
solo a quest'ordine è possibile esprimere davvero il principio di
induzione. Il principio d'induzione infatti fa riferimento alla totalità
dei sottoinsiemi dei numeri naturali i quali sono una quantità più
che numerabile mentre lo schema d'assiomi che lo esprime al primo
ordine fa riferimento solo alla totalità dei sottoinsiemi rappresentabili
che in quanto quantità numerabile non esauriscono la totalità dei
sottoinsiemi. Uno dei corollari del teorema di Godel è che la logica
del secondo ordine è incompleta e una teoria formale la cui logica
è incompleta non può essere deduttivamente completa quindi l'aritmetica
del secondo ordine da questo punto di vista subisce una sorte analoga
a quella del primo. 

L'aritmetica del secondo ordine però a differenza di quella del primo
ammette una completezza descrittiva della struttura vista la categoricità
della teoria e come già accennato la cosa implica la completezza semantica.
Quindi seppur per ogni formula $\varphi$ del linguaggio della teoria
al secondo ordine non è possibile che $\vdash\varphi$ o $\vdash\neg\varphi$
succede però che $\models\varphi$ o $\models\neg\varphi$, infatti
se non si desse il caso che per ogni formula $\varphi$ del linguaggio
$\models\varphi$ o $\models\neg\varphi$ allora esisterebbero due
modelli M' e M'' tali che M'$\models\neg\varphi$ e M''$\models\varphi$
ma ciò va contro la categoricità della teoria%
\footnote{L'argomento è valido solo classicamente facendo riferimanto alla reductio
ad absurdum e alla legge logica valida solo classicamente secondo
la quale $\neg\forall xAx$ $\rightarrow$ $\exists x\neg Ax$.%
}.

Alla luce di queste considerazioni sembra allora che il primo teorema
di incompletezza non comporti dei limiti nel metodo assiomatico ma
solo dei limiti nello sviluppo della matematica da un punto di vista
sintattico, infatti un matematico può lavorare senza problemi al secondo
ordine dove i modelli dell'aritmetica formale sono uno solo a meno
di isomorfismi, e quindi non si hanno problemi di caratterizzabilità
della struttura, ma un sistema formale non può lavorare ad un tale
ordine in quanto i sistemi deduttivi del secondo ordine non sono completi.

Il primo dei due aspetti sopra indicati come fondanti il programma
di Hilbert sembra perciò indebolito dal risultato godeliano che segna
uno scarto tra l'approcciò proof-teoretico e quello model-teoretico
e la necessità di distinzione tra il concetto di provabilità in un
sistema formale e quello di verità in un modello.


\subsection{Secondo teorema di incompletezza}

Il secondo teorema di incompletezza di Godel afferma che la consistenza
dell'aritmetica formale non è dimostrabile all'interno del sistema
formale stesso e più in generale che ogni teoria formale altrettano
espressiva non ha mezzi sufficienti per esprimere la propria coerenza.
Da ciò segue che una dimostrazione di coerenza di un tale sistema
deve necessariamente fare ricorso a qualche principio non contenuto
in esso. Quindi se si cerca una dimostrazione finitaria della coerenza
di tale teoria, si deve estendere l'ambito della matematica finitaria
oltre ciò che è esprimibile nell'aritmetica formale e questo effettivamente
è stato fatto da Gerhard Gentzen che trovò nel 1936 una dimostrazione
di coerenza per l'aritmetica formale che fa uso dell'induzione fino
ad un particolare numero ordinale transfinito, usualmente chiamato
epsilon-zero. Se poi dall'aritmetica si passa all'analisi matematica,
cioè ad una teoria formale per i numeri reali, la dimostrazione di
coerenza deve attendere il 1970 e richiede l'induzione fino ad un
ordinale infinitamente più grande di epsilon-zero%
\footnote{Sambin, Alla ricerca della certezza perduta, pg.24%
}. A questo punto è lecito chiedersi se una tale matematica possa ancora
ritenersi finitaria e se rientri ancora nell'ambito di una matematica
contenutisticamente sicura e che non necessiti di fondazione.

A tutto ciò si aggiunge una difficoltà ben più grande riguardante
questa volta la teoria ZF. Come per le teorie dei numeri sopra viste
la teoria ZF non è in grado di dimostare la propria coerenza quindi
per farlo abbiamo bisogno di un principio non dimostrabile in ZF stessa.
Ora, da un lato è del tutto irragionevole ritenere che la matematica
finitaria non sia contenuta nella teoria degli insiemi nella quale
è di fatto contenuta tutta la matematica d'oggi e dall'altro, fatto
ancora più rilevante, è attualmente del tutto inimmaginabile un principio
matematico che trascenda la teoria degli insiemi e quindi risulta
proprio impossibile allo stato attuale fornire una prova di coerenza
di tale teoria%
\footnote{Sambin, Alla ricerca della certezza perduta, pg.24%
}.

Con il secondo teorema di incompletezza di Godel viene così a cadere
il secondo assunto fondante il programma di Hilbert: la matematica
non può fondarsi su se stessa, essa è destinata a poggiare su qualcosa
di cui si può leggittimante dubitare. Concretamente, tramite la teoria
ZF è possibile provare la coerenza di tutte le teoria matematiche
tranne ZF stessa.


\section{Stato dell'arte}

Passata la grande stagione della logica che va dai primi del Novecento
alla fine degli anni '40, l'indagine sui fondamenti della matematica
si è affievolita e tecnicizzata in modo da divenire una vera e propria
disciplina matematica che fa ampio uso degli strumenti sviluppati
dalla logica matematica. Il problema dei fondamenti si è così svincolato
da questioni più generali di filosofia della matematica, quali questioni
epistemologiche ed ontologiche, ma in alcuni casi non può evitare
riferimenti ad esse. La natura di questa disciplina quindi fa continuamente
riferimento da una lato alla logica e dall'altro alla pratica matematica
avendo però il tipico andamento di una discplina filosofica(tante
teorie discordanti, dicussioni, conflitti...), tutto ciò la rende
unica nel suo genere.

Come già accennato solo due delle tre grandi famiglie di proposte
fondazionali hanno dato vita ad una concreta pratica matematica e
quindi le prospettive fondazionali più studiate sono quelle riferentisi
a questi due indirizzi, che sono rispettivamente il formalismo e l'intuizionismo.
Queste due prospettive sono molto diverse ed a tratti divergenti,
all'interno di ogni prospettiva inoltre ci sono numerose proposte
particolari.

L'intuizionismo è caratterizzato dall' adozione di una logica non
classica, la logica intuizionistica per l'appunto, e dall'uso esclusivo
di tecniche di dimostrazione costruttive. La semantica della logica
intuizionista da un punto di vista algebrico può essere vista come
una generalizzazione della semantica classica dal momento che la struttura
algebrica cui si riferisce è una generica algebra di Heyting e che
ogni algebra di Boole è un'algebra di Heyting. Diverse sono le leggi
logiche classiche non valide intuizionisticamente ad esempio la legge
di reductio ad absurdum, il terzo esclusoe e la doppia negazione.
Queste restrizioni rientrano in un quadro coerente e chiaro la qual
cosa rende la logica intuizionista una logica molto elegante e plausibile.
L'adozione di metodi costruttivi rende poi la matematica intuizionista
molto più concreta di quella classica e passibile di numerose e interessanti
applicazioni in ambito informatico. Tra le discipline logico matematiche
c'è sicuramente una predilezione per la proof-theory e la teoria della
calcolabilità.

Il formalismo invece è caratterizzato dall'adozione della logica classica
e dalla liceità di qualsiasi tecnica dimostrativa plausibile senza
interesse per la sua effettività. Il formalismo, che è la forma in
cui si presenta la matematica classica nel panorama contemporaneo,
ha una natura prevalentemente semantica e i suoi fondamenti sono da
rintracciare nella teoria degli insiemi e nella teoria dei modelli.
I concetti e gli strumenti dimostrativi di ogni teoria in questa matematica
infatti fanno riferimento alla teoria degli insiemi ZF o a qualche
versione della teoria degli insiemi ad essa equivalente e la coerenza(relativa)
delle sue teorie è dimostrata tramite tecniche model-teoretiche%
\footnote{Ad esempio tramite la teoria degli insiemi si costruisce un modello
dell'aritmetica e con esso si costruiscono i reali passando per interi
e razionali arrivando così a dimostrare la coerenza dell'analisi.%
}. Il formalismo ha portato ad uno sviluppo senza precedenti della
matematica classica ed alla creazione di teorie notevolmente eleganti
soprattutto grazie allo sviluppo di tecniche algebriche sempre più
raffinate. 

Per dare un giudizio generale sui due approcci fondazionali si potrebbe
dire che l'intuizionismo ha fondamenta più salde ma è più debole(accetta
meno cose), mentre il formalismo ha fondamenta più fragili ma è più
forte(accetta più cose). In fin dei conti tutto dipende dall'accettare
di abitare una grande casa sospesa su un filo che non si sa dove poggia
o prendere calce e mattoni e mettersi a costruire una casetta su un
terreno sicuro e tranquillo.
