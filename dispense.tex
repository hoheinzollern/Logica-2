\documentclass[10pt,a4paper]{amsbook}

\usepackage{ucs}
\usepackage[utf8x]{inputenc}
\usepackage{amsmath}
\usepackage{amsfonts}
\usepackage{amssymb}
\usepackage{amsthm}
\usepackage[italian]{babel}
\usepackage{fontenc}
\usepackage[all]{xy}
\usepackage{graphicx}

\usepackage{hyperref}
\hypersetup{%
    colorlinks=true, linktocpage=true, pdfstartpage=3, pdfstartview=FitV,%
    breaklinks=true, pdfpagemode=UseNone, pageanchor=true,
    pdfpagemode=UseOutlines, plainpages=false, bookmarksnumbered,
    bookmarksopen=true, bookmarksopenlevel=1, hypertexnames=true,
    pdfhighlight=/O,%hyperfootnotes=true,%nesting=true,%frenchlinks,%
    %urlcolor=webbrown, linkcolor=RoyalBlue, citecolor=webgreen,
    %pagecolor=RoyalBlue,%
    % uncomment the following line if you want to have black links (e.g., for
    % printing)
    %urlcolor=Black, linkcolor=Black, citecolor=Black, pagecolor=Black,%
    pdftitle={Dispense di Logica 2},%
    pdfauthor={},%
    pdfsubject={},%
    pdfkeywords={},%
    pdfcreator={pdfLaTeX},%
    pdfproducer={LaTeX with hyperref and classicthesis}%
}

\usepackage{tikz}
\usetikzlibrary{shapes,arrows}
% Define block styles
\tikzstyle{decision} = [diamond, draw, fill=white, 
    text width=3.5em, text badly centered, node distance=2cm, inner sep=0pt]
\tikzstyle{block} = [rectangle, draw, fill=white, 
    text width=5em, text centered, minimum height=2em]
\tikzstyle{line} = [draw, -latex']
\tikzstyle{noblock}=[rectangle,
    text width=5em, text centered, minimum height=2em]

\newcommand{\roundentry}[1]{*++[o][F-] {#1}} 
\newtheorem{esempio}{Esempio}[chapter]
\newtheorem{oss}{Osservazione}[chapter]
\newtheorem{prop}{Proposizione}[chapter]
\newtheorem{defi}{Definizione}[chapter]
\newtheorem{thm}{Teorema}[chapter]
\newtheorem{extra}{Esercizio}
\newtheorem{nota}{Nota}
\newtheorem{programmi}{Definizione}[section]
\newtheorem{progabaco}[programmi]{Teorema}


\newenvironment{mylisting}
{\begin{list}{}{\setlength{\leftmargin}{1em}}\item\bfseries}
{\end{list}}

\newenvironment{mytinylisting}
{\begin{list}{}{\setlength{\leftmargin}{1em}}\item\tiny\bfseries}
{\end{list}}

\title{Dispense di Logica 2}
\author{}
\date{}

\begin{document}

\maketitle

\tableofcontents

\newcommand{\bew}{\mathsf{Bew}}
% per scrivere eqz più velocemente
\newcommand{\eq}[1]{\begin{equation} #1 \end{equation}}
% per fare le parentesi squadrate
\newcommand{\gdnum}[1]{\ulcorner #1 \urcorner}
% per scrivere pi velocemente un'eqz e darle un label
\newcommand{\eqconnome}[2]{\begin{equation} \label{eq:#2} #1 \end{equation}}

\chapter{Macchine di Turing}

\section{Introduzione}
Intuitivamente diciamo che una funzione \`e calcolabile se esiste un insieme
finito di istruzioni che ci dica esattamente cosa dobbiamo fare, senza lasciare
nulla di sottointeso ed ambiguo. Possiamo parlare quindi di una procedura
effettiva (meccanica) proprio perch\`e la sua esecuzione deve dipendere
unicamente dagli input e dalle regole che applico (fattori come il tempo, lo
spazio, il singolo esecutore non devono influire sul risultato) e quindi
possiamo immaginare tale procedura eseguibile da una macchina ipotetica.

Pensiamo ad una macchina come ad un sistema che possiede certe abilit\`a, \`e
quindi in grado di fare certe cose. Ci concentreremo sulle procedure effettive
che potremo scrivere con una certa macchina e che tipo di funzioni potremo
rappresentare.

\section{Definizione di Macchina di Turing}
Il logico inglese Alan Turing nel suo articolo del 1936 era interessato alla
questione di cosa volesse dire calcolabile e descrisse quindi dei dispositivi
astratti conosciuti come Macchine di Turing. Prima di tutto vediamo pi\`u in
dettaglio come Turing descrive una macchina.

Turing stesso paragon\`o una macchina al comportamento di un essere umano che
sta eseguendo un calcolo.  Immaginiamo perci\'o un uomo seduto ad una scrivania
munito di carta e penna: immaginiamo quindi quest'uomo come una testina di
controllo che possa scrivere simboli di un certo alfabeto su di un ipotetico
foglio di carta che noi chiameremo nastro.

Poich\'e non \`e importante il numero di fogli a disposizione dell'uomo,
supporremo la carta come una risorsa infinita: questo si traduce nel fatto che
il nastro su cui la testina di controllo scrive sia infinito in entrambe le
direzioni.

In secondo luogo, l'uomo pu\`o ricordare ed ha memoria finita, quindi dotiamo la
testina di un numero finito di stati di memoria. Infine, la scrittura su di un
pezzo di carta comporta l'avanzamento del processo di scrittura; parallelamente
la scrittura su nastro di un simbolo comporter\`a l'avanzamento verso destra o
verso sinistra della testina medesima.

Dopo queste analogie possiamo riassumere la definizione di Macchina di Turing in
questo modo.

Una macchina di Turing (MdT) \`e definita da un \textsl{insieme di
  regole} che definiscono il comportamento della macchina su un nastro
di input-output (lettura e scrittura). Il nastro pu\`o essere
immaginato come un nastro di lunghezza infinita diviso in quadratini
dette \textsl{celle}. Ogni cella contiene un simbolo $S_i$ di un certo
alfabeto oppure \`e vuota e questo viene rappresentato dal carattere
Blank ($B$). La macchina ha una testina che si sposta lungo il nastro
iniziando dalla cella che contiene il simbolo pi\`u a sinistra e
analizzando una cella alla volta: i simboli devono essere posizionati
in celle vicine e si possono concatenare anche pi\`u input dividendoli
al pi\`u da un carattere B di Blank. I caratteri di Blank si
susseguono anche al di fuori della sequenza di simboli dati in input,
occupando tutta la lunghezza infinita del nastro.  La testina \`e
dotata di un numero finito di stati di memoria $q_i$. Lo stato ha la
funzione di un'etichetta: ad uno stato $q_i$ corrisponde un'operazione
se viene letto un simbolo $S_j$ e un'altra operazione se viene letto
il simbolo $B$. Inoltre a seconda dello stato in cui la testina si
trova, sa che simboli ha letto prima.

\begin{figure}[htbp!]
\centering
\includegraphics[scale=0.8]{img/Nastro.jpg}
\end{figure}

La macchina legge, scrive oppure cancella i simboli nelle celle del nastro
seguendo l'insieme di regole o quintuple,  che ne definiscono il comportamento.
Tali regole sono di questo tipo:

\vspace{0.3cm}
(\textsl{stato-interno-corrente}, \textsl{simbolo-letto},
\textsl{simbolo-scritto}, \textsl{prossimo-stato-interno}, \textsl{direzione}).
\vspace{0.3cm}

Quindi ad ogni passo, la macchina legge un simbolo sul nastro e in accordo al
suo \textsl{stato interno corrente}:

\begin{enumerate}
\item scrive un simbolo sul nastro
\item decide il suo prossimo stato interno, e decide se spostare la testina a
sinistra (L) o a destra (R) di una posizione.
\end{enumerate}
\vspace{0.5cm}

Diamo di seguito un esempio per queste istruzioni:

\begin{itemize}
\item La quintupla $<q_{i}S_{j}S_{k}Rq_{l}>$ indica che se la macchina
  si trova nello stato interno $q_{i}$ e legge sul nastro il simbolo
  $S_{j}$, allora lo sostituisce con $S_{k}$ e sposta la testina di
  lettura di una cella verso destra passando allo stato $q_{l}$
\item La quintupla $<q_{i}S_{j}S_{k}Lq_{l}>$ indica che se la macchina
  si trova nello stato $q_{i}$ e legge sul nastro il simbolo $S_{j}$,
  allora lo sostituisce con $S_{k}$ e sposta la testina di lettura di
  una cella verso sinistra lungo il nastro passando nello stato
  $q_{l}$.\\ Notiamo che i simboli $S_{j}$ e $S_{k}$ possono
  coincidere (cio\`e non sovrascrive nulla $<q_{i}S_{j}S_{j}Rq_{l}>$),
  cos\`i come $q_{i}$ pu\`o essere uguale a $q_{l}$ (ossia rimane
  nello stesso stato $<q_{i}S_{j}S_{j}Rq_{i}>$).
\item La scrittura del simbolo $B$ di Blank corrisponde a cancellare
  il contenuto di una cella. Ad esempio la quintupla
  $<q_{i}S_{j}BRq_{l}>$ indica che se la macchina si trova nello stato
  $q_{i}$ e legge $S_{j}$, allora cancella il simbolo mettendo il
  blank e sposta la testina di lettura di una cella verso destra
  passando allo stato $q_{l}$.
\end{itemize}

Noi tratteremo, in accordo col fatto che il nostro ``uomo anonim\`o''
non pu\`o scegliere cosa fare ma solo eseguire regole date. Diamo
quindi la seguente definizione:

\begin{defi}
Una Macchina di Turing si dice \underline{deterministica} se dato uno
stato $q_{i}$ e un simbolo $S_{j}$ c'\`e \textsl{al pi\`u} una
quintupla che inizi per $<q_{i},S_{j}...>$ cio\`e la testina
trovandosi in $q_{i}$ e leggendo $S_{j}$ non pu\`o scegliere cosa fare
ma solo eseguire l'unica istruzione corrispondente, \textsl{se c'\`e},
altrimenti significa che la macchina si arresta.
\end{defi}

Inoltre \`e possibile descrivere il nastro in ogni fase della
computazione che fa la macchina. Introduciamo quindi la seguente
definizione:

\begin{defi}
Una \underline{descrizione istantanea} $\alpha$ di una macchina di
Turing $\tau$ \`e un'espressione contenente esattamente un $q_{i}$ (lo
stato), i simboli sul nastro, e la cella in cui si trova la testina al
momento.
\end{defi}

Quindi una descrizione istantanea (ID) $\alpha$ \`e una stringa della
forma $$S_1S_2S_3.....S_{i-1}qS_{i}S_{i+1}......S_{n}$$ dove

\begin{enumerate}
\item q \`e lo stato della macchina di Turing;
\item $S_1S_2S_3.........S_{n}$ \`e la porzione non-blank del nastro;
\item La testina \`e sopra il simbolo i-esimo.
\end{enumerate}
\vspace{0.5cm}

Vediamo ora come una Macchina di Turing effettua i suoi
calcoli. Inizialmente il nastro contiene una sequenza finita di
simboli, detta \textsl{sequenza di ingresso}. Assumiamo che all'inizio
la testina della macchina $\tau$ sia posizionata sulla prima cella di
input, cio\`e sulla prima cella in cui c'\`e un simbolo e diciamo che
$\tau$ si trova nello stato $q_0$. La testina vede un simbolo $S_i$ e
allora parte: cerca tra le quintuple che forniscono le varie
istruzioni quella che comincia per $q_0$ e ha come secondo elemento il
simbolo letto, del tipo $<q_0,S_i..>$. A questo punto vede quale
istruzione deve eseguire (se sovrascrivere un altro simbolo o se
cancellare), in quale direzione andare e in quale stato portarsi, e
cos\`i via. Se non c'\`e una tale quintupla, la macchina si arresta.

Osserviamo che Turing propose l'idea di tale macchina che fosse capace
di eseguire ogni tipo di calcolo su numeri e simboli. Noi per
semplicit\`a ci riconduciamo a calcoli su numeri naturali e vedremo
che ci possiamo ridurre a considerare un alfabeto costituito
unicamente dai simboli 1 e B: $$\Sigma=\left\{1,B\right\}$$ e quindi
con tale alfabeto riusciremo a rappresentare tutti i numeri
naturali. Notiamo infatti che \`e del tutto equivalente considerare un
alfabeto con un simbolo o con n simboli ed i calcolatori, essendo
rappresentazioni di Macchine di Turing ed usando l'alfabeto binario lo
confermano.

Introduciamo ora una notazione che ci permetter\`a di rappresentare i
numeri naturali e quindi gli input da dare alle macchine (compreso lo
0). Per come abbiamo definito la macchina di Turing, risulta che
abbiamo a disposizione l'unico simbolo di input $1$ per rappresentare
ogni naturale o n-upla di naturali che siano nel dominio di una
funzione.

\begin{itemize}
\item preso un numero $n\in {N}$, ad esso associamo un'espressione
$$\overline{n}=1^{n+1}=\underbrace{11...1}_{n+1\ volte}\quad\mbox{.}$$
  Ad esempio $\overline{3}=1111$.
\item in maniera del tutto analoga, a una k-tupla $(n_{1}...n_{k})$ di
  interi non negativi associamo l'espressione
  $\overline{(n_{1}...n_{k})}=\overline{n_{1}}B...B\overline{n_{k}}$. Per
  esempio $$\overline{(1,4,0)}=\overline{1}B\overline{4}B\overline{0}=
  11B11111B1\quad\mbox {.}$$ In questo caso si vede come si procede
  per separare gli input, cio\`e inserendo un $B$.
\end{itemize}

Se invece con questa notazione abbiamo rappresentato nel nastro un
determinato numero indichiamo con $<\overline{n}>$ il numero naturale
da esso rappresentato, ossia $<\overline{n}>=n$.\\ Quindi gli input
che compariranno nella sequenza d'ingresso sono della forma
$$...BB\overline{n_{1}}B\overline{n_{2}}B...B\overline{n_{k}}BB...$$
dove $\overline{n_1},...,\overline{n_k}$ sono i numeri naturali nella
notazione data sopra. Converremo che ogni blocco rappresentante un
argomento \`e separato dagli altri da un solo $B$; in questo modo la
macchina sa dove inizia e termina l'input (se si incontrano due blank
consecutivi, la testina non \`e pi\`u collocata su una delle celle di
input).  La macchina di Turing su un dato input produce un output. Noi
accetteremo output della forma $$...BB\overline{n_r}BB...$$ ove
$\overline{n_r}$ \`e il risultato della computazione. Dunque in output
avremo il nastro contenente un blocco unico contenente tanti $1$
quanti sono necessari per rappresentare il risultato, in modo tale che
anche in questo caso la macchina sa dove comincia e finisce l'output,
e quindi si noti che, una volta calcolata la soluzione, sar\`a
necessario anche cancellare gli input.

\section{Varianti della Macchina di Turing}
Ci sono molteplici varianti della macchina di Turing e seppur non
entrando qui nel dettaglio si pu\`o mostrare che queste hanno la
stessa portata computazionale della macchina di Turing che abbiamo
definito all'inizio.
\begin{itemize}
\item Macchina di Turing Multinastro: invece di esservi un singolo
  nastro, vi sono k nastri indipendenti e k testine di
  lettura-scrittura, una per ogni nastro.
\item Macchina di Turing con movimento arbitrario della testina di
  lettura: si pu\`o modificare la definizione di una macchina di
  Turing in modo che la testina di lettura si possa spostare ogni
  volta di un numero arbitrario di celle
\item Macchina di Turing con nastro bidimensionale: la macchina opera
  su un nastro bidimensionale, e sono permessi spostamenti verso
  sinistra, destra, l'alto, il basso.
\item Macchina di Turing non deterministica: si distingue da quella
  deterministica definita in precedenza per il fatto che, in presenza
  di un determinato stato e di un determinato carattere letto, essa
  permette la presenza di pi\`u quintuple che iniziano con quello
  stato e quel carattere letto.
\end{itemize}


\chapter{Macchine di Turing e Funzioni Calcolabili}

\section{Cosa pu\`o Essere Calcolato}

Le macchine di Turing sono strumenti molto potenti. Per una grande
quantit\`a di problemi di calcolabilit\`a, \`e possibile costruire una
macchina di Turing che sia in grado di eseguire il calcolo
desiderato. Per esempio, abbiamo visto che \`e possibile ideare una
macchina di Turing per eseguire le operazioni aritmetiche di base sui
numeri naturali.

\subsection{Numeri Calcolabili}

Il lavoro originale di Turing trattava i numeri calcolabili. Un numero
si dice Turing-calcolabile se esiste una macchina di Turing che inizia
la computazione su un nastro vuoto e calcola un numero arbitrario, con
una certa approssimazione. Le macchine di Turing possono calcolare
tutti i numeri algebrici (radici di equazioni con coefficienti
algebrici) e molte costanti matematiche, come ad esempio $\textit{e}$
(costante di Nepero) e $\pi$.

\subsection{Funzioni Calcolabili}

Come abbiamo visto in precedenza, le macchine di Turing possono fare
molto pi\`u che calcolare semplici numeri. Tra le altre cose possono
calcolare funzioni numeriche del tipo $f:N^{n}->N$, come ad esempio si
possono scrivere macchine per calcolare la somma, la moltiplicazione,
la sottrazione tra naturali, l'esponenziale, il fattoriale e molte
altre.

Usando una convenzione per rappresentare i simboli di verit\`a TRUE e
FALSE, rappresentando, per esempio, il valore TRUE come una sequenza
di due ``1'' e il valore FALSE come un singolo ``1'', si possono
trattare anche funzioni binarie, ovvero del tipo
$f:N^{n}->\{0,1\}$. Con tali funzioni \`e possibile calcolare la
funzione caratteristica di un predicato. E', inoltre, possibile
combinare i risultati di tali funzioni con gli usuali operatori
booleani: AND, NOT, OR e in aggiunta IF-THEN-ELSE.

Difatti, le funzioni Turing-calcolabili coincidono con le funzioni
ricorsive, di cui tratteremo pi\`u avanti.

\subsection{Macchina di Turing Universale}

Il pi\`u importante e suggestivo risultato del lavoro di Turing
riguarda il fatto che, per come sono state definite le macchine di
Turing, \`e possibile definire la \textit{Macchina di Turing
  Universale} (UTM).

In teoria della computazione si dice macchina di Turing Universale
$T_U$ una macchina di Turing capace di simulare le evoluzioni di ogni
macchina di Turing. Tale macchina ha permesso di dare una risposta
negativa al problema della decidibilit\`a posto da David Hilbert nel
1928. Brevemente, il problema posto da Hilbert riguarda la
possibilit\`a o meno di poter definire un metodo matematico o un
processo per decidere tutti i possibili problemi matematici.

Essenzialmente la UTM \`e in grado di simulare il comportamento di
qualunque macchina di Turing (compresa se stessa!). Un modo per
pensare alla UTM potrebbe essere quello di immaginarla come un
interprete di un linguaggio di programmazione. Quando all'interprete
(concretamente \`e esso stesso un programma) viene dato in input un
programma, questo comincia ad eseguire le istruzioni del programma
dato e, quindi, si comporta proprio come se fosse lo stesso programma,
dato in input, ad essere eseguito.

La macchina di Turing Universale richiede in input una coppia di
numeri naturali (n,m), dove il primo \`e un indice che identifica una
specifica macchina di Turing $T_n$ (in grado di computare una data
funzione f). Mentre, il secondo indice rappresenta l'input sul quale
eseguire la macchina $T_n$.

Per poter definire una macchina del tipo descritto, \`e prima
necessario ideare un modo per rappresentare una macchina di Turing sul
nastro della UTM. E' possibile associare in modo effettivo ad ogni
macchina di Turing un numero naturale che la identifichi
univocamente. L'idea \`e quella di avere una funzione, preferibilmente
biunivoca, che mappi l'insieme delle macchine di Turing, diciamo T,
nell'insieme dei numeri naturali $f:T->N$. A tale scopo, tenendo
presente che le macchine di Turnig sono definite da una quintupla, si
pu\`o prima definire una codifica per i singoli elementi di una
quintupla e in seguito per ogni possibile quintupla. Esistono molti
modi possibili per ottenere questa codifica effettiva. Il paragrafo
\ref{codeffettiva} che segue mostra a titolo di esempio una di queste
possibili codifiche.

L'esecuzione della macchina universale $T_U(n,m)$ corrisponde quindi
ad eseguire la macchina $T_n(m)$.

\begin{center}
$T_U(n,m) = T_n(m)$
\end{center}

\vspace{10pt}

La macchina $T_U$ opera con un nastro che inizialmente in una sua
prima sezione contiene la codifica delle istruzioni della macchina
$T_n$ seguita, dopo un demarcatore particolare bene individuabile,
dallo stesso input m.

E' interesante osservare che la macchina di Turing Universale \`e in
grado di simulare anche la propria evoluzione mentre procede a
simulare una qualsiasi macchina $T$.

I computer programmabili sono effettivamente delle macchine di Turing
universali, poich\`e sono in grado di simulare l'esecuzione di ogni
macchina di Turing (se vengono intese come algoritmi che calcolano
funzioni) tramite l'esecuzione di un programma, che pu\`o essere visto
come un processo di calcolo da eseguire.

\subsection{Codifica Effettiva delle Macchine di Turing}\label{codeffettiva}

Di seguito viene riportato un'esempio di codifica effettiva per le
macchine di Turing. Con tale codifica \`e possibile rappresentare una
qualunque macchina di Turing sul nastro della macchina di Turing
universale.\\

Ogni quintupla nella definizione di una macchina sar\`a codificata
come una sequenza di cinque blocchi di ``1'', separati da un singolo
``0''.
\begin{enumerate}
  \item Il primo blocco di ``1'' sar\`a la codifica del numero dello
    stato corrente, usando la convezione sopra definita per la
    rappresentazione dei numeri naturali sul nastro (n+1 ``1''
    rappresentano il numero n).
  \item Il secondo blocco di ``1'' sar\`a la codifica del simbolo
    corrente (simbolo che si trova sotto la testina). La codifica del
    simbolo corrente dipende dall'alfabeto e dalla notazione per i
    caratteri usata.``1''
  \item Il terzo blocco di ``1'' sar\`a la codifica dell'eventuale
    simbolo da scrivere in posizione sul nastro sottostante la
    testina.
  \item Il quarto elemento rappresenter\`a l'azione che deve essere
    eseguita dalla testina, e ci sono tre possibilit\`a: con un blocco
    di un solo ``1'' si rappresanta lo spostamento nullo, con un
    blocco di due ``1'' lo spostamento a sinistra (L) e con tre ``1''
    lo spostamento a destra (R).
  \item Infine, il quinto elemento della tuple rappresenter\`a il
    numero del nuovo stato in cui la macchina si trover\`a dopo
    l'esecuzione dell'azione richiesta, rappresentato anche esso
    mediante la notazione unaria.
\end{enumerate}

Per codicifare, in fine, l'intera macchina, basta semplicemente
scrivere sul nastro tutte le tuple che caratterizzano la macchina,
separate tra loro da un carattere di \textit{Blank}.

Se si utilizza il carattere ``0'' al posto del carattere
\textit{Blank}, si ottiene una sequenza di ``0'' e di ``1''. Tale
sequenza pu\`o essere interpretata come un numero naturale espresso in
rappresetntazione binaria. Se tale numero viene espresso con la
notazione $\overline{n}=1^{n+1}$, risulta effettivamente possibile
dare in input alla macchina di Turing universale una coppia di numeri
naturali espressi secondo questa notazione.


\section{Cosa non pu\`o Essere Calcolato}
Un altro sorprendente risultato presentato dall'articolo di Turing del
1936 riguarda l'esistenza di problemi ben definiti che non possono
essere risolti da nessuna procedura computazionale. Se questi problemi
sono formulati come formule, possiamo riferirci a queste funzioni come
funzioni non calcolabili; se questi problemi vengono espressi come
predicati, si dice problemi indecidibili. Usando il concetto della
macchina astratta di turing, possiamo dire che una funzione non \`e
calcolabile se non esistono macchine di Turing che sono in grado di
calcolare tale funzione.

Per mostrare l'esitenza di funzioni non calcolabili dalle macchine di
Turing bisogna ragionare sulla cardinalit\`a dell'insieme delle
macchine di Turing e confrontarla con la cardinalit\`a dell'insieme
delle funzioni del tipo $f:N->N$ (con uno sforzo minimo ci si pu\`o
portare al caso pi\`u generale di funzioni $f:N^n->N$).

\begin{center}
$F = \{ f | f \textrm{ \`e una funzione del tipo } f:N->N \}.$
\end{center}

\vspace{10pt}

\begin{center}
$T = \{ t | t \textrm{ \`e una macchina di Turing} \}.$
\end{center}

\vspace{10pt}

Poich\`e \`e possibile, come mostrato in precedenza, costruire una
codifica effettiva delle macchine di Turing, cio\`e costruire una
funzione che associ ad ogni TM un numero naturale unico, l'insieme
delle TM \`e enumerabile, ovvero la sua cardinalit\`a \`e pari a $|N|$
(cardinalit\`a dei numeri naturali). Tuttavia, l'insieme di tutte le
funzioni $f:N->N$ \`e dimostrato non essere enumerabile. Per questo
motivo, possiamo concludere che tutte le macchine di Turing sono in
grado di calcolare solo un sottoinsieme di tutte le funzioni
definibili.

Ci sono molti problemi legati ai programmi che si possono dimostrare
non decidibili. Poich\`e le macchine di Turing sono molto semplici se
comparate con i computer come uso pratico, risulta concettualmente
pi\`u facile provare l'impossobilit\`a di alcuni risultati per le
macchine di Turing. Dando un definizione formale di non banale per una
propriet\`a, H.G. Rice riusc\`i, addirittura, a dimostrare il seguente
teorema:

\begin{thm}
Ogni propriet\`a non banale dei programmi \`e indecidibile.
\end{thm}

E' da notare che dimostrando che un problema \`e non decidibile, non
significa che non pu\`o essere deciso solo in alcuni casi, bens\`i che
\`e impossibile costruire una soluzione \textit{per ogni} possibile
input.



%%%%%%%%%%%%%%%%%%%%%%%%%%%%%%%%%%%%%%%%%%%%%%%%%ESEMPI%%%%%%%%%%%%%%%%%%%%%%%%%%%%%%%%%

\chapter{Macchine di Turing: Esempi}


\section{Esempi di Macchine di Turing}
Diamo ora qualche esempio di semplici macchine di Turing.\\

\begin{itemize}
\item Sia $\Sigma=\left\{0,1\right\}$ l'alfabeto di simboli in
  input. \\\underline{Scopo della macchina:} data una stringa di 0 e 1
  scambia le occorrenze di 1 con 0 e viceversa, riposizionandosi sulla
  cella precedente a quella iniziale.\\ \underline{Passi:} leggere i
  simboli sul nastro e ogni volta che si trova 1 si scrive 0 e
  viceversa. Quando la sequenza di 0 e 1 è finita la testina torna
  sulla cella precedente alla prima significativa.\\

Nastro in input:$$...BB100101BB....$$\\

Sequenza di quintuple:
\begin{eqnarray*}
&<q_0,0,1,R,q_0>&\\
&<q_0,1,0,R,q_0>&\mbox{(scambia 0 e 1)}\\
&<q_0,B,B,L,q_1>&\\
&<q_1,0,0,L,q_1>&\\
&<q_1,1,1,L,q_1>&\mbox{(si riposiziona sulla cella precedente alla prima significativa)}\\
\end{eqnarray*}
Quindi la macchina di Turing per la sequenza di ingresso $...BB100101BB....$ esegue la seguente computazione:
\begin{eqnarray*}
&...BBq_0100101BB...\\
&...BB0q_000101BB...\\
&...BB01q_00101BB...\\
&...BB011q_0101BB...\\
&...BB0110q_001BB...\\
&...BB01101q_01BB...\\
&...BB011010q_0BB...\\
&...BB01101q_10BB...\\
&...\\
&...Bq_1B011010BB...\\
\end{eqnarray*}
Notiamo che l'ultima descrizione istantanea \`e tale da essere
terminale. Infatti arrivata a questo punto la macchina si arresta in
quanto nell'insieme di quintuple che regolano il suo comportamento non
c'\`e una quintupla del tipo $<q_1,B....>$.\\
\vspace{0.3 cm} Nastro in output:$$...BB011010BB....$$ Quindi dato
l'output vediamo che tale macchina sostituisce, nella sequenza data in
input, le occorrenze di 1 con 0 e viceversa.\\
\vspace{0.2cm}
Diagramma:
\begin{figure}[hbtp!]
\hspace{3 cm} \entrymodifiers={++[o][F-]} \xymatrix@=40pt{q_{0}
  \ar@(dr,dl)[]^{1|0 \ R} \ar@(ul,ur)[]^{1|0 \ R} \ar@<2pt>[r]^{B|B
    \ L} & q_{1}\ar@(dr,dl) []^{0|0 \ L} \ar@(ul,ur)[]^{1|1 \ L}\\}
\end{figure}

\item Sia $\Sigma=\left\{1,B\right\}$ l'alfabeto di simboli in
  input. \\\underline{Scopo della macchina:} calcolare il successore
  di un numero naturale dato in input ovvero data una sequenza di
  $n+1$ uni, che rappresenta il numero naturale n, aggiunge alla fine
  della sequenza un altro uno($f(x)=x+1$).\\\underline{Passi:} Essendo
  la testina posizionata nella cella con il primo 1 della sequenza si
  sposta fino al primo B a destra della sequenza. Scrive 1 e torna
  alla posizione di partenza.\\ Nastro in input: $x=2$ rappresentato
  da 3 uni $$BBB111BBB$$ Sequenza di quintuple:
\begin{eqnarray*}
&<q_0,1,1,R,q_0>&\mbox{(scorre tutti gli uni della sequenza)}\\
&<q_0,B,1,L,q_1>&\mbox{(sostituisce il primo B trovato a destra della sequenza con un 1)}\\
&<q_1,1,1,L,q_1>&\\
&<q_1,B,B,R,q_{2}>&\mbox{(si riposiziona sulla prima cella significativa del nastro)}\\
\end{eqnarray*}
Quindi la macchina di Turing per la sequenza di ingresso
$...BB111BB....$ esegue la seguente computazione:
\begin{eqnarray*}
&...BBq_0111BB...\\
&...BB1q_011BB...\\
&...BB11q_01BB...\\
&...BB111q_0BB...\\
&...BB11q_011BB...\\
&...\\
&...Bq_0B1111BB\\
\end{eqnarray*}
Nastro in output: $f(2)=3$ rappresentato da 4 uni $$...BB1111BB....$$
Diagramma:
\begin{figure}[hbtp!]
\hspace{3 cm} \entrymodifiers={++[o][F-]} \xymatrix@=40pt{q_{0}
  \ar@(dr,dl)[]^{1|1 \ R} \ar@<2pt>[r]^{B|1 \ L} & q_{1}\ar@(dr,dl)
      []^{1|1 \ L} \ar@<2pt>[r]^{B|B \ R} & q_{2}}\\
\end{figure}
\begin{oss}
Per calcolare il successore, invece di aggiungere un uno alla fine
della sequenza di uni data in input, si poteva aggiungerlo nella cella
precedente alla prima cella significativa del nastro. In tal caso le
quintuple si sarebbero ridotte a due e precisamente:
\begin{eqnarray*}
&<q_0,1,1,L,q_1>&\mbox{(scorre tutti gli uni della sequenza)}\\
&<q_1,B,1,R,q_2>&\mbox{(sostituisce il primo B trovato a destra della sequenza con un 1)}\\
\end{eqnarray*}
\end{oss}
\item Sia $\Sigma=\left\{1,B\right\}$ l'alfabeto di simboli in
  input.\\\underline{Scopo della macchina:} data in input una sequenza
  (anche di pi\`u input basta siano separati al pi\`u da un B) sposta
  tutta la parte significativa del nastro di una cella verso
  destra.\\ \underline{Passi:} cancellare il primo 1 che si trova, poi
  scorrere la parte significativa del nastro fino al primo B e
  sostituirlo con un 1. Se c'\`e un altro input procede
  analogamente.\\ Nastro in input:$$...BB11B1BB....$$ Sequenza di
  quintuple:
\begin{eqnarray*}
&<q_0,1,B,R,q_1>&\mbox{(cancella il primo 1)}\\
&<q_1,1,1,R,q_1>&\mbox{(scorre la sequenza di 1)}\\
&<q_1,B,1,R,q_2>&\mbox{(sostituisce il primo B con un 1, cos\`i sposta il primo blocco)}\\
&<q_2,1,B,R,q_1>&\mbox{(cancella il primo 1 del secondo input e procede come prima)}\\
&<q_2,B,B,L,q_3>&\\
&<q_3,1,1,L,q_3>&\\
&<q_3,B,B,L,q_4>&\\
&<q_4,1,1,L,q_3>&\\
&<q_4,B,B,R,q_5>&\\
&<q_5,B,B,R,q_6>&\mbox{(si riposiziona sulla prima cella significativa del nastro)}\\
\end{eqnarray*}
Quindi la macchina di Turing per la sequenza di ingresso $...BB11B1BB....$ esegue la seguente computazione:
\begin{eqnarray*}
&...BBq_011B1BB...\\
&...BBBq_11B1BB...\\
&...BBB1q_1B1BB...\\
&...BBB11q_21BB...\\
&...BBB11Bq_1BB...\\
&...BBB11B1q_2BB...\\
&...BBB11Bq_31BB...\\
&...BB011q_3B1BB...\\
&...BBB1q_41B1BB...\\
&...BBBq_311B1BB...\\
&...BBq_3B11B1BB...\\
&...Bq_4BB11B1BB...\\
&...BBq_5B11B1BB...\\
&...BBBq_611B1BB...\\
\end{eqnarray*}
Nastro in output: tutta la sequenza \`e stata spostata di una cella
verso destra $$...BBB11B1BB....$$ Diagramma:
\begin{figure}[hbtp!]
\hspace{3 cm} \entrymodifiers={++[o][F-]} \xymatrix@=40pt{q_{0}
  \ar[d]^{1|B \ R}\\ q_{1}\ar@(dr,dl) []^{1|1 \ R} \ar@<2pt>[r]^{B|1
    \ R} & q_{2}\ar@(d,dr)[l]^{1|B \ R} \ar@<2pt>[r]^{B|B \ L} &
  q_{3}\ar@(dr,dl) []^{1|1 \ L}\ar@<2pt>[r]^{B|B \ L } &
  q_{4}\ar@(d,dr)[l]^{1|1 \ L} \ar@<2pt>[r]^{B|B \ R} & q_{5}
  \ar@<2pt>[r]^{B|B \ R} & q_{6}}\\
\end{figure}

\end{itemize}

\section{Alcune operazioni aritmetiche}
Mostriamo ora alcune macchine di Turing pi\`u complesse, in grado di
calcolare alcune operzioni aritmetiche.\\

\begin{esempio}[Sottrazione]
Vogliamo calcolare la differenza di due numeri naturali non
negativi: $$g(x,\ y)=x-y.$$ L'idea \`e quella di cancellare di volta
in volta un 1 all'inizio e un 1 alla fine della sequenza in input
finch\`e uno dei due blocchi non si svuota e aggiunge un 1 per avere
l'output nella forma standard.\newline Notare che nei naturali $x-y$,
con $x<y$, non \`e definita e per questo la nostra macchina di Turing
per la sottrazione potrebbe andare in loop, ossia non fermarsi
mai. Questo \`e esattamente un esempio di funzione semicalcolabile.\\

Definizione della macchina:
\begin{eqnarray*}
&<q_{0},1,B,R,q_{1}>&\mbox{(cancella l'1 pi\`u a sinistra)}\\
&<q_{1},1,1,R,q_{1}>&\\
&<q_{1},B,B,R,q_{2}>&\mbox{(trova il blank che separa i blocchi)}\\
&<q_{2},1,1,R,q_{2}>&\\
&<q_{2},B,B,L,q_{3}>&\mbox{(trova l'estremit\`a destra)}\\
&<q_{3},1,B,L,q_{4}>&\mbox{(cancella l'1 pi\`u a destra)}\\
&<q_{4},1,1,L,q_{5}>&\\
&<q_{4},B,1,L,q_{9}>&\mbox{(se $y$ \`e stato consumato torno sulla prima cella altrimenti continua)}\\
&<q_{5},1,1,L,q_{5}>&\\
&<q_{5},B,B,L,q_{6}>&\mbox{(trova il blank che separa i blocchi)}\\
&<q_{6},1,1,L,q_{7}>&\\
&<q_{6},B,B,R,q_{8}>&\mbox{(se $x$ \`e stato consumato torna all'inizio senn\`o continua)}\\
&<q_{7},1,1,L,q_{7}>&\\
&<q_{7},B,B,R,q_{0}>&\mbox{(trova l'estremit\`a sinistra e torna in $q_{0}$)}\\
&<q_{8},1,1,L,q_{8}>&\\
&<q_{8},B,B,R,q_{8}>&\mbox{(computa un ciclo infinito se $x< y$)}\\
&<q_{9},1,1,L,q_{9}>&\\
&<q_{9},B,B,R,q_{10}>&\mbox{(torno sulla prima cella significativa del nastro)}\\
\end{eqnarray*}

Grafico:
\begin{figure}[hbtp]
\resizebox{\columnwidth}{!}{%
\hspace{0cm} \xymatrix@=40pt{\roundentry{q_{0}} \ar[r]^{1|BR} &
  \roundentry{q_{1}} \ar@(dr,dl)[]^{1|1R} \ar[r]^{B|BR} &
  \roundentry{q_{2}} \ar@(dr,dl)[]^{1|1R} \ar[r]^{B|BL} &
  \roundentry{q_{3}} \ar@(dr,dl)[]^{1|BL} \ar[r]^{1|BL} &
  \roundentry{q_{4}} \ar[r]^{B|1L} \ar[d]^{1|1L} & \roundentry{q_{9}}
  \ar@(dr,dl)[]^{1|1L} \ar[r]^{B|BR} & \roundentry{q_{10}} \\ &&&&
  \roundentry{q_{5}} \ar@(dr,dl)[]^{1|1L} \ar[r]^{B|BL} &
  \roundentry{q_{6}} \ar[r]^{B|BR} \ar[d]^{1|1L} & \roundentry{q_{8}}
  \ar@(dr,dl)[]^{1|1L} \ar@(ru,rd)[]^{B|BR} \\ &&&&&
  \roundentry{q_{7}} \ar@(dr,dl)[]^{1|1L} \ar `l[lllll] ^{B|BR}
             [llllluu] }}
\end{figure}

Diamo ora qualche esempio di computazione e calcoliamo
quindi $$\overline{m_{1}}-\overline{m_{2}}\mbox{,}\quad
\overline{m_{1}}\mbox{,} \overline{m_{2}}\in \mbox{\upshape N.}$$ La
configurazione iniziale con cui la macchina parte \`e:
$$...BBq_{0}1^{m_{1}+1}B1^{m_{2}+1}BB...$$

Ora la testina si trova in $q_{0}$ e legge un 1, quindi deve cancellarlo. Dunque abbiamo:
$m_{1}\geq m_{2}$. Chiamiamo $k=m_{1} - m_{2}$.\\
\begin{eqnarray*}
&...BBBq_{1}1^{m_{1}}B1^{m_{2}+1}BB...&\mbox{(ho cancellato il primo 1)}\\
&...BBB1q_{1}1^{m_{1}-1}B1^{m_{2}+1}BB...&\\
&.........&\\
&...BBB1^{m_{1}}q_{1}B1^{m_{2}+1}BB...&\\
&...BBB1^{m_{1}}Bq_{2}1^{m_{2}+1}BB...&\mbox{(ho trovato la B che separa i blocchi)}\\
&...BBB1^{m_{1}}B1q_{2}1^{m_{2}}BB...&\\
&.........&\\
&...BBB1^{m_{1}}B1^{m_{2}+1}q_{2}BB...&\\
\end{eqnarray*}
\begin{eqnarray*}
&...BBB1^{m_{1}}B1^{m_{2}}q_{3}1BB...&\mbox{(ho trovato l'ultimo 1)}\\
&...BBB1^{m_{1}}B1^{m_{2}}q_{4}BBB...&\\
&...BBB1^{m_{1}}B1^{m_{2}-1}q_{4}1BBB...&\mbox{(ho cancellato l'ultimo 1)}\\
&...BBB1^{m_{1}}B1^{m_{2}-2}q_{5}11BBB...&\\
&...BBB1^{m_{1}}B1^{m_{2}-3}q_{5}111BBB...&\\
&.........&\\
&...BBB1^{m_{1}}q_{5}B1^{m_{2}}BBB...&\\
&...BBB1^{m_{1}-1}q_{6}1B1^{m_{2}}BBB...&\mbox{(ho trovato la B tra i due blocchi)}\\
&...BBB1^{m_{1}-2}q_{7}11B1^{m_{2}}BBB...&\\
&...BBB1^{m_{1}-3}q_{7}111B1^{m_{2}}BBB...&\\
&.........&\\
&...BBq_{7}B1^{m_{1}}B1^{m_{2}}BBB...&\\
&...BBBq_{0}1^{m_{1}}B1^{m_{2}}BBB...&\mbox{(sono tornato all'inizio e ricomincio)}\\
&.........\qquad\qquad\qquad&[*]\\
&...BBB^{m_{2}}q_{0}11^{k}B1B^{m_{2}}BB...&\\
&...BBB^{m_{2}}q_{0}B1^{k}B1B^{m_{2}}BB...&\\
&...BBB^{m_{2}+1}q_{1}1^{k}B1B^{m_{2}}BB...&\\
&...BBB^{m_{2}+1}1q_{1}1^{k-1}B1B^{m_{2}}BB...&\\
&.........&\\
&...BBB^{m_{2}+1}1^{k}q_{1}B1B^{m_{2}}BB...&\\
&...BBB^{m_{2}+1}1^{k}Bq_{2}1B^{m_{2}}BB...&\\
&...BBB^{m_{2}+1}1^{k}B1q_{2}B^{m_{2}}BB...&\\
&...BBB^{m_{2}+1}1^{k}Bq_{3}1B^{m_{2}}BB...&\\
&...BBB^{m_{2}+1}1^{k}Bq_{3}BB^{m_{2}}BB...&\mbox{($m_{2}$ \`e stato consumato)}\\
&...BBB^{m_{2}+1}1^{k}q_{4}BBB^{m_{2}}BB...&\\
&...BBB^{m_{2}+1}1^{k}q_{9}1BB^{m_{2}}BB...&\\
&...BBB^{m_{2}+1}1^{k-1}q_{9}11BB^{m_{2}}BB...&\\
&.........&\\
&...BBB^{m_{2}}q_{9}B1^{k+1}BB^{m_{2}}BB...&\\
&...BBB^{m_{2}+1}q_{10}1^{k+1}BB^{m_{2}}BB...&\mbox{(sono tornato all'inizio)}\\
\end{eqnarray*}

Il risultato della computazione \`e esattamente $k$.\\

($m_{1}<m_{2}$). Chiamiamo $k=m_{2}-m_{1}$. Partendo dalla stessa
configurazione iniziale del caso precedente, ottengo una serie di
descrizioni istantanee come sopra, fino a [*]. Riprendiamo la nostra
computazione proprio da questo punto. Arriveremo a:\\
\begin{eqnarray*}
&...BBB^{m_{1}}q_{0}1B1^{k}1B^{m_{1}}BB...&\\
&...BBB^{m_{1}+1}q_{1}B1^{k}1B^{m_{1}}BB...&\\
&...BBB^{m_{1}+2}q_{2}1^{k}1B^{m_{1}}BB...&\\
&...BBB^{m_{1}+2}1q_{2}1^{k-1}1B^{m_{1}}BB...&\\
&.........&\\
&...BBB^{m_{1}+2}1^{k}1q_{2}B^{m_{1}}BB...&\\
&...BBB^{m_{1}+2}1^{k}q_{3}1B^{m_{1}}BB...&\\
&...BBB^{m_{1}+2}1^{k-1}q_{4}1B^{m_{1}+1}BB...&\\
&...BBB^{m_{1}+2}1^{k-2}q_{5}11B^{m_{1}+1}BB...&\\
&...BBB^{m_{1}+2}1^{k-3}q_{5}111B^{m_{1}+1}BB...&\\
&.........&\\
&...BBB^{m_{1}+1}q_{5}B1^{k}B^{m_{1}+1}BB...&\mbox{($m_{1}$ \`e stato svuotato!)}\\
&...BBB^{m_{1}}q_{6}BB1^{k}B^{m_{1}+1}BB...&\\
&...BBB^{m_{1}+1}q_{8}B1^{k}B^{m_{1}+1}BB...& \qquad[\dag]\\
&...BBB^{m_{1}+2}q_{8}1^{k}B^{m_{1}+1}BB...&\\
&...BBB^{m_{1}+1}q_{8}B1^{k}B^{m_{1}+1}BB...& \qquad[\dag]\\
\end{eqnarray*}

Notare che le due descrizione istantanee contrassegnate con $[\dag]$
sono le stesse, questo significa che siamo entrati in un ciclo
infinito. Questo dipende dal fatto che nei naturali, la sottrazione
tra due numeri $x$ e $y$, con $x<y$, non \`e definita.\\
\end{esempio}


\begin{esempio}[Sottrazione Adeguata]
Possiamo evitare il problema del ciclo infinito nell'esempio
precedente definendo una \textsl{sottrazione adeguata}:\\
\begin{center}
$g(x,y)=x\dot{-} y=
\left\{ \begin{array}{ll}
x-y &  x \geq y\\
0 & \textrm{altrimenti}
\end{array} \right.$
\end{center}
Una macchina di Turing che computi tale funzione \`e molto simile a
quella dell'esempio precedente: la definizione della macchina a
partire da tutti gli stati diversi da $q_{8}$ \`e la stessa e abbiamo
in aggiunta le seguenti mosse e stati:
\begin{eqnarray*}
&<q_{8},1,1,L,q_{8}>&\\
&<q_{8},B,B,R,q_{11}>&\\
&<q_{11},1,B,R,q_{8}>&\\
&<q_{11},B,1,R,q_{12}>&\\
\end{eqnarray*}
Il grafico \`e come quello per lo stato $q_{8}$:
\begin{figure}[hbtp]
\hspace{0cm} \entrymodifiers={++[o][F-]} \xymatrix@=40pt{ q_{6}
  \ar[r]^{B|BR} & q_{8} \ar@(dr,dl)[]^{1|1L} \ar@<2pt>[r]^{B|BR} &
  q_{11} \ar@(d,dr)[l]^{B|BR} \ar[r]^{B|1-} & q_{12}}\end{figure}\par
In questa maniera, se $x\geq y$ la macchina funziona come nell'esempio
precedente, altrimenti cancella tutto tranne un $1$ (ricordare che
$<1>=<\overline{0}>=0$).\\

E' da notare che non \`e possibile, in generale, modificare ogni
funzione parziale in modo tale da permetterle di esecuire gli stessi
calcoli e allo stesso tempo renderla totale, cos\`i da evitare
potenziali esecuzioni infinite. In questo caso, infatti, abbiamo
potuto modificare la funzione in modo tale da farle ritornare un
valore particolare nel caso in cui la funzione originale non sia
definita sull'input dato. In questo modo la macchina di Turing che la
calcola termina sempre su qualunque input, ma questo tipo di modifica
in generale non \`e sempre possibile.\\
\end{esempio}

\begin{esempio}[Prodotto]
Calcoliamo ora il prodotto di due numeri\\ \underline{naturali non
  negativi}: $g(m,n) = m * n$\\

L'idea in questo caso \`e quella di sommare $m$ a s\`e stesso $n$
volte; per fare questo necessario usare una cella del nastro come
segnaposto per capire a che punto della moltiplicazione si \`e
arrivati. Ovviamente ci sono dei casi particolari in cui uno dei due
numeri o eventualmente entrambi sono nulli ed essi vanno analizzati
singolarmente.\\

In questo esempio una volta arrivati alla conclusione della
computazione, invece di tornare all'inizio del nastro, si fa andare la
macchina allo stato $q_{42}$, per cui si suppone che non ci siano
istruzioni.\\

Definizione della macchina:\\
\begin{eqnarray*}
&<q_{0},1,B,R,q_{1}>&\mbox{(cancello il primo ``1'')}\\
&<q_{1},B,B,R,q_{2}>&\mbox{(caso in cui ho uno zero nel primo input)}\\
&<q_{2},1,B,R,q_{2}>&\\
&<q_{2},B,1,L,q_{42}>&\mbox{(il secondo input viene cancellato e quando si arriva alla fine }\\
&&\mbox{si trasporma il primo ``B'' che si trova in ``1'' per indicare lo ``0'')}\\
&<q_{1},1,B,R,q_{3}>&\mbox{(cancello il secondo ``1'' del primo input)}\\
&<q_{3},1,1,R,q_{3}>&\\
&<q_{3},B,B,R,q_{4}>&\mbox{(scorro tutto il primo input)}\\
&<q_{4},1,1,R,q_{5}>&\mbox{(sono arrivato all'inizio del secondo input quindi}\\
&&\mbox{ trovo sicuramente un ``1'')}\\
&<q_{5},B,B,L,q_{6}>&\mbox{(caso in cui il secondo input \`e zero)}\\
&<q_{6},1,B,L,q_{6}>&\\
&<q_{6},B,B,L,q_{7}>&\\
&<q_{7},1,B,L,q_{7}>&\\
&<q_{7},B,1,L,q_{42}>&\mbox{(anche qui l'``1'' che si aggiunge alla fine \`e per indicare lo zero)}\\
&<q_{5},1,B,R,q_{8}>&\mbox{(caso in cui entrambi gli input sono diversi da zero; l'``1'' che}\\
&&\mbox{ si cancella in questo momento \`e usato come segnaposto)}\\
&<q_{8},1,1,R,q_{8}>&\\
\end{eqnarray*}
\begin{eqnarray*}
&<q_{8},B,B,R,q_{9}>&\mbox{(scorro il secondo input)}\\
&<q_{9},1,1,R,q_{9}>&\mbox{(scorro il risultato)}\\
&<q_{9},B,1,L,q_{10}>&\\
&<q_{10},1,1,L,q_{10}>&\\
&<q_{10},B,B,L,q_{11}>&\mbox{(aggiungo un ``1'' alla fine e poi torno indietro fino a quando arrivo }\\
&&\mbox{al ``B'' che separa il secondo input dal risultato)}\\
&<q_{11},B,1,L,q_{12}>&\mbox{(caso in cui sono gi\`a arrivato alla fine del secondo input)}\\
&<q_{12},1,1,L,q_{12}>&\\
&<q_{12},B,B,L,q_{13}>&\\
&<q_{13},1,1,L,q_{14}>&\mbox{(se sono arrivato alla fine del secondo input tolgo il segnaposto e}\\
&&\mbox{ torno all'inizio del primo input)}\\
&<q_{13},B,B,R,q_{1}>&\mbox{(caso in cui sono ho gi\`a finito)}\\
&<q_{14},1,1,L,q_{14}>&\\
&<q_{14},B,B,R,q_{1}>&\\
&<q_{11},1,1,L,q_{15}>&\\
&<q_{15},1,1,L,q_{15}>&\\
&<q_{15},B,1,R,q_{16}>&\mbox{(ho trovato il segnaposto)}\\
&<q_{16},1,B,R,q_{8}>&\mbox{(aggiorno il segnaposto)}\\
&<q_{16},B,B,R,q_{9}>&\mbox{(quando arrivo alla fine del secondo input) }\\
\end{eqnarray*}
\end{esempio}


\chapter{Macchine a Registri}


\section{Premessa}
Lo studio delle macchine di Turing ci ha fatto capire che il metodo di
dare istruzioni per definirle risulta piuttosto complesso da
implementare e lontano dal nostro modo di ragionare (si veda per
esempio la macchina di Turing che calcola il prodotto).

Per questo introduciamo le macchine a registri, ossia delle
meta-macchine che lavorano con un linguaggio di secondo livello, che
\`e pi\`u complesso ma che permette di strutturare le istruzioni da
dare alla macchina di Turing in modo pi\`u facile e lineare. \`E
simile alla situazione di un linguaggio di programmazione di vario
livello: il linguaggio di programmazione che una macchina pu\`o
eseguire si chiama Assembler ed \`e poco pi\`u che una stringa di 0 e
1. Inizialmente era questo il linguaggio usato per la
programmazione. Successivamente sono stati inventati dei linguaggi di
pi\`u alto livello: si tratta di linguaggi pi\`u comprensibili che non
vengono eseguiti direttamente, ma vengono trasformati da un
metaprogramma in istruzioni del linguaggio Assembler.  Allo stesso
modo qui abbandoniamo il linguaggio macchina, che \`e quello delle
macchine di Turing (elementare, ma niente affatto banale, perch\'e
questo resta quello che sapremo fare), e adesso diamo delle istruzioni
di pi\`u alto livello che si possono leggere come abbreviazioni di
istruzioni di linguaggio. Possiamo quindi dire che:
\begin{thm}
Ogni funzione calcolabile a registri \`e Turing-cal\-co\-la\-bi\-le.
\end{thm}
\textsc{Dimostrazione}: \`e ovvio perch\'e, per le motivazioni scritte
sopra, ciascuna operazione eseguibile con una macchina a registri non
\`e altro che un insieme strutturato di istruzioni che definiscono una
macchina di Turing che calcola la medesima operazione.\\

Per proseguire con la definizione di macchina a registri, dobbiamo
introdurre dei registri di memoria dove inserire dei valori numerici
che si possano estrarre quando servono durante il calcolo. Inoltre
vogliamo che tali registri di memoria siano accessibili in modo
diretto.

Il fatto che una macchina di Turing lavori come un nastro permette di
registrare dei dati, ma non di accedervi in modo diretto: per esempio
se abbiamo \( k \) ingressi possiamo usare i blocchi dal \( k + 1
\)-esimo in poi come registri di memoria, ma in questo modo per andare
a vedere cosa c'\`e nel \( k + 1 \)-esimo posto dobbiamo spostarci sul
nastro, memorizzando dove eravamo e tornare indietro con inserita,
dentro lo stato interno, l'informazione; tutto ci\`o risulta molto
complicato.  Ora invece vediamo che \`e possibile definire un
linguaggio che faccia queste cose in modo automatico.

I registri contengono sempre un numero intero positivo. Un aspetto
importante \`e che non ci sono limitazioni n\'e alla grandezza dei
registri, cio\'e possiamo mettere in un registro numeri grandi a
piacere, n\'e al numero di registri che vogliamo usare. Potremo quindi
avere \( k \) registri che contengono gli input, un registro per l'
output e tutti i registri ausiliari che ci occorrono durante l'
operazione per inserire i risultati intermedi. Le operazioni permesse
sono:
\begin{itemize}
\item sommare 1 al contenuto del registro \( R_i \);
\item sottrarre 1 al contenuto del registro \( R_i \); vogliamo poter
  applicare tale operazione in ogni caso, quindi se c'\`e qualcosa nel
  registro si toglie 1 altrimenti il registro resta vuoto;
\item avere accesso diretto al registro \( R_i \).
\end{itemize}

Indicando con un \( n \) cerchiato il registro \( R_n \), scriviamo:
\begin{itemize}
\item \( n+ \) cerchiato per indicare l'operazione ``aggiungi 1 al registro \( R_n \)''
\item  \( n- \) cerchiato, per indicare l'operazione ``se il registro \( R_n \) non \`e vuoto
sottrai 1''.
\end{itemize}

Utiliz\-zia\-mo poi delle frecce che ci dicono come proseguire dopo
aver eseguito un' operazione:
\begin{itemize}
\item dal nodo \( n+ \) abbiamo una sola freccia che si legge
  ``aggiungi 1 al registro \( R_n \) e prosegui''
\item dal nodo \( n- \) partono due frecce e si legge ``se il registro
  \( R_n \) \`e vuoto prosegui a destra, se non \`e vuoto sottrai 1 e
  prosegui sotto''. Si noti che questo corrisponde all'aver introdotto
  l'operatore condizionale ``\emph{if-then-else}''.
\end{itemize}

\section{Esempi di Macchine a Registri}
Presentiamo ora una serie di esempi che aiuteranno a familiarizzare
con le macchine a registri.

\begin{esempio}[Proiezione \( p_n^i(x_1, \ldots, x_n)=x_i \).]

    Abbiamo supposto che, dati \( n \) registri, abbiamo accesso
    diretto \mbox{all' \( i \)-esimo} registro. Vogliamo l' output nel
    registro specifico da noi scelto (\textbf{ad esempio}) \( R_{3}
    \). Basta quindi accedere all' \( i \)-esimo registro e svuotarlo
    in \( R_{3} \).

    \begin{figure}[hbtp]
    \hspace{0cm}
    \xymatrix{
    \roundentry{i-} \ar@(ul,ul)[] \ar[d] \ar[r] & stop \\
    \roundentry{3+} \ar@(d,dl)[u]}
    \caption{Macchina a registri che calcola la proiezione sull' \( i \)-esimo registro}
    \end{figure}

\end{esempio}

\begin{esempio}[Successore]

    Partiamo con l' input in \( R_1 \) e vogliamo l' output in \(
    R_{3} \). Iniziamo aggiungendo 1 in \( R_{3} \) e poi svuotiamo \(
    R_1 \) in \( R_{3} \). (Si noti che non \`e detto che la macchina
    debba partire da \( R_1 \).)

    \begin{figure}[hbtp]
    \hspace{0cm}
    \xymatrix{
    \roundentry{3+} \ar@(ul,ul)[] \ar[d] \\
    \roundentry{1-} \ar@(d,dl)[u] \ar[r] & stop}
    \caption{Macchina a registri che calcola il successore di un numero dato in input}
    \end{figure}

\end{esempio}

\begin{esempio}[Differenza] \ \\
    \begin{center}
    $n \dot{-} m=
    \left\{ \begin{array}{ll}
    n-m & \textrm{se } m \leq n\\
    0 & \textrm{altrimenti}
    \end{array} \right.$
    \end{center}


    \ \\ I registri \( R_1 \) e \( R_2 \) contengono rispettivamente
    \( n \) e \( m \) e vogliamo l' output in \( R_{3} \) (che
    supponiamo inizialmente vuoto). Iniziamo togliendo 1 da \( R_2 \),
    poi sottraiamo 1 da \( R_1 \); continuiamo cos\`i fino a che \(
    R_2 \) \`e vuoto. A questo punto il contenuto di \( R_1 \) \`e $n
    - m$ e lo svuotiamo in \( R_{3} \). Se invece \( R_1 \) resta
    vuoto prima che si sia svuotato \( R_2 \) (cio\`e $n < m$) ci
    fermiamo (\( R_{3} \) \`e rimasto vuoto).

    \begin{figure}[hbtp]
    \hspace{0cm}
    \xymatrix{
      \roundentry{1-} \ar[r] \ar@(dr,d)[d] & stop \\
     \roundentry{2-} \ar@(ul,ul)[] \ar[u] \ar[r] & \roundentry{1-} \ar[d] \ar[r] & stop \\
     & \roundentry{3+} \ar@(d,dl)[u]
    }

   %  \includegraphics[scale=0.5]{diffeLabel.png}
    \caption{Macchina a registri che calcola la differenza $n \dot{-} m$}
    \end{figure}

\end{esempio}

\begin{esempio}[Somma \( m \) con \( n \), senza dimenticare \( m \).]
    Prendiamo tre registri:
    \begin{itemize}
    \item un registro \( R_1 \) dentro cui ci sar\`a \( m \);
    \item un registro \( R_2 \) dentro cui ci sar\`a \( n \);
    \item un registro ausiliario \( R_3 \), che inizialmente sar\`a vuoto;
    \end{itemize}
    Vogliamo ottenere l' output in \( R_2 \) e che \( m \) rimanga in
    \( R_1 \). L' operazione da eseguire \`e la seguente:

    Partiamo dal registro \( R_1 \): togliamo 1 dal contenuto di \(
    R_1 \), aggiungiamo 1 a quello di \( R_2 \) e poi aggiungiamo 1 in
    \( R_3 \). Ripetiamo queste operazioni fino a che \( R_1 \) rimane
    vuoto; ci\`o significa che abbiamo ``svuotato'' \( R_1 \) in \(
    R_2 \) e quindi in \( R_2 \) alla fine c' \`e \( m+n \), mentre in
    \( R_3 \) c' \`e \( m \), perch\'e ogni volta che passiamo in \(
    R_2 \) (\( m \) volte), aggiungiamo 1. Poi svuotiamo \( R_3 \) in
    \( R_1 \), cio\`e togliamo 1 da \( R_3 \) e aggiungiamo 1 in \(
    R_1 \) fino a quando \( R_3 \) rimane vuoto. A questo punto
    abbiamo finito.

    \begin{figure}[hbtp]
    \hspace{0cm}
    \xymatrix{
    \roundentry{1-} \ar@(ul,ul)[] \ar[r] \ar[d] & \roundentry{3-} \ar[d] \ar[r] & stop \\
    \roundentry{2+} \ar[d] & \roundentry{1+} \ar@(d,dl)[u] \\
    \roundentry{3+} \ar@(d,dl)[uu]
    }
    \caption{Macchina a registri che somma \( m \) con \( n \), senza dimenticare \( m \)}
    \end{figure}

\end{esempio}

\`E evidente come questo sistema sia molto pi\`u comprensibile rispetto a quello usato nelle macchine di Turing.

\begin{esempio}[Moltiplica \( m \) per \( n \)]
    Partiamo mettendo \( m \) in \( R_1 \) ed \( n \) in \( R_2 \);
    vogliamo il risultato in \( R_3 \). Usiamo \( R_4 \) come registro
    ausiliario.

    L'idea \`e sommare \( n \) con s\`e stesso \( m \) volte, quindi
    ogni volta che togliamo 1 da \( R_1 \) dobbiamo aggiungere al
    risultato \( n \). Per fare ci\`o cominciamo col sottrarre 1 dal
    registro \( R_1 \), svuotiamo il registro \( R_2 \), che
    all'inizio contiene \( n \), nel registro \( R_3 \) e allo stesso
    tempo mettiamo la stessa quantit\`a \( n \) nel registro \( R_4
    \); quando il registro \( R_2 \) \`e svuotato, rimettiamo il
    contenuto del registro \( R_4 \) nel registro \( R_2 \). A questo
    punto nel registro \( R_2 \) c'\`e \( n \), in \( R_4 \) c'\`e 0
    ed in \( R_3 \) c'\`e \( n \). Quindi ritorniamo al registro \(
    R_1 \) e sottraiamo un altro 1; ripetiamo l'operazione, e cio\'e
    la quantit\`a \( n \) del registro \( R_2 \) va svuotata nel
    registro \( R_3 \), quindi \( R_3 \) avr\`a \( n \) + \( n \) e \(
    R_4 \) avr\`a \( n \). Quando \( R_2 \) \`e vuoto risvuotiamo \(
    R_4 \) in \( R_2 \) e ripetiamo questo ciclo finch\'e \( R_1 \)
    non \`e vuoto. Quando quest'ultimo \`e vuoto ci fermiamo e
    leggiamo il risultato in \( R_3 \).

    \begin{figure}[hbtp]
    \hspace{0cm}
    \xymatrix{
    \roundentry{1-} \ar@(ul,ul)[] \ar[d] \ar[r] & stop \\
    \roundentry{2-} \ar[r] \ar[d] & \roundentry{4-} \ar[d] \ar@(r,r)[ul] \\
    \roundentry{3+} \ar[d] & \roundentry{2+} \ar@(d,dl)[u] \\
    \roundentry{4+} \ar@(d,dl)[uu]}
    \caption{Macchina a registri che moltiplica \( m \) per \( n \)}
    \end{figure}

\end{esempio}

\begin{esempio}[Composizione]

    Supponiamo che $ f $ e $ g $ siano due funzioni calcolabili a
    registri e vogliamo dimostrare che $ g\circ f $ \`e calcolabile a
    registri. Esistono quindi due macchine a registri \( M_f \) e \(
    M_g \) che calcolano rispettivamente \( f \) e \( g \) partendo
    dal registro \( R_1 \) e restituendo l' output in \( R_42
    \). Af\mbox{}f\mbox{}inch\'e non ci siano ambiguit\`a bisogna
    cambiare i nomi ai registri di \( M_g \), e quindi costruire una
    macchina a registri \( M_{g'} \) i cui re\-gi\-stri siano \(
    R_{n+1} \), \ldots, \( R_{n+42} \), con \( n \) maggiore del
    numero di registri di \( M_f \). A questo punto facciamo partire
    \( M_f \), questa restituisce l' output in \( R_{42} \), che
    svuotiamo in \( R_{n+1} \). Basta poi far partire \( M_{g'} \) e
    scaricare l' output di \( M_{g'} \) in \( R_{42} \).


    \begin{figure}[hbpt]
    \hspace{0cm} \xymatrix{ \framebox[1cm]{\( M_f \)} \ar@(ul,ul)[]
      \ar@(d,l)[r] & \roundentry{42-} \ar[d] \ar[r] &
      \framebox[1cm]{\( M_{g'} \)} \ar@(d,l)[r] &
      \roundentry{\ (n+42)\ -} \ar[d] \ar[r] & stop \\ &
      \roundentry{\ (n+1)\ +} \ar@(d,dl)[u] & & \roundentry{42+}
      \ar@(d,dl)[u] }
     \caption{Macchina a registri che calcola la composizione di due funzioni}
    \end{figure}
\end{esempio}

Ora che abbiamo visto come funzionano le macchine a registri, diamo
una definizione formale di "funzione calcolabile a re\-gi\-stri".

\begin{thm}
Una funzione \( f \) (di \( n \) variabili) \`e calcolabile a registri
se esiste una macchina a registri che, mettendo gli input \( x_1,
\ldots,\\ x_n \) nei registri pref\mbox{}issati \( R_1, \ldots, R_n
\), d\`a output \( f(x_1, \ldots, x_n) \) per ogni \( x_1, \ldots, x_n
\).
\end{thm}

\chapter{Funzioni Ricorsive}
\label{lemma diagonalizzazione}

Lo scopo di questo capitolo \`e quello di dimostrare l'equivalenza tra
i concetti di calcolabilit\`a con registri e calcolabilit\`a con
programmi.  Iniziamo dunque a definire i programmi.

\section{I programmi}
Nella sezione precedente abbiamo visto come il contenuto dei registri,
nelle macchine a registri, possa essere modificato attraverso
determinate \emph{istruzioni} che la macchina \`e in grado di
riconoscere. Una sequenza di queste semplici istruzioni costituisce un
\emph{programma}.

Supponendo $x_i$, $x_j$ e $x_k$ essere nomi di variabili, tali
istruzioni possono essere di quattro tipi:

\begin{enumerate}
\item $x_j \leftarrow 0$ svuota $x_j$ del suo contenuto
\item $x_j \leftarrow x_i $ assegna il contenuto di $x_i$ a $x_j$
\item $x_j \leftarrow x_j+1$ aumenta il contenuto di $x_j$ di uno e
  riassegna il risultato come valore a $x_j$
\item $\textrm{if }x_i \neq0$ then goto $\alpha$ else goto $\beta$ con
  $\alpha$ e $\beta$ istruzioni del programma
\end{enumerate}
	
\begin{nota}
Nella (\emph{4}) \`e possibile sostituire alla condizione $x_i \neq0$
una qualunque formula atomica di un linguaggio che usi le relazioni
$=,\geq,<$ e le operazioni $+,-,successore,\ldots $ e le loro
composizioni.
\end{nota}

\begin{extra}
Si scriva un programma in cui sia presente un'istruzione del tipo
(\emph{4}) avente per condizione $x_i\neq x_j$. (\emph{Suggerimento}:
si pensi ad una funzione in $(x_i,\ldots ,x_j)$ il cui valore sia
diverso da $0$ quando $x_i\neq x_j$)
\end{extra}	

\begin{extra}
Mostrare, con qualche esempio, che \`e possibile sostituire alla
condizione della (\emph{4}) una qualsiasi espressione di
calcolo \newline booleano classico.
\end{extra}	

Per eseguire una \emph{computazione} la macchina deve essere
fornita di un \emph{programma} $P$ e una \emph{configurazione
  iniziale}, che pu\`o essere, ad esempio, una sequenza $a_1, a_2,
...$ di numeri naturali memorizzati nelle variabili $x_i, x_j,
...$. Supponiamo che $P$ sia composto da $s$ istruzioni $I_1, I_2,...,
I_s$.

Allora la macchina inizia la computazione osservando $I_1$, poi $I_2$,
e cos\`i via, a meno che non si incontri un'istruzione di \emph{goto}
che fa saltare la computazione in una delle $s$ istruzioni. La
computazione si ferma quando, e solo quando non c'\`e una prossima
istruzione. Possiamo illustrare questo attraverso un esempio.
	
\begin{esempio} Consideriamo il seguente programma con $s=6$ istruzioni:
\begin{mylisting}
$I_1$: $\textrm{if }x_1 > x_2$ then goto $I_2$ else goto $I_5$
  \\ $I_2$: $x_2 \leftarrow x_2+1$\\ $I_3$: $x_3 \leftarrow
  x_3+1$\\ $I_4$: $\textrm{if }x_1 \neq x_2$ then goto $I_2$ else goto
  $I_5$\\ $I_5$: $x_0 \leftarrow x_3$\\
\end{mylisting}
E con la seguente configurazione iniziale:
\begin{mylisting}
$x_0 = 0, x_1 = 9, x_2 = 7, x_3 = 0, ...$
\end{mylisting}
Durante il calcolo le variabili vengono modificate come segue:
\begin{mylisting}	
$x_0 = 0, x_1 = 9, x_2 = 7, x_3 = 0, ...$ ($I_1: x_1 > x_2$)\\ $x_0 =
  0, x_1 = 9, x_2 = 8, x_3 = 0, ...$ ($I_2$)\\ $x_0 = 0, x_1 = 9, x_2
  = 8, x_3 = 1, ...$ ($I_3$)\\ $x_0 = 0, x_1 = 9, x_2 = 8, x_3 = 1,
  ...$ ($I_4: x_1 \neq x_2$)\\ $x_0 = 0, x_1 = 9, x_2 = 9, x_3 = 1,
  ...$ ($I_2$)\\ $x_0 = 0, x_1 = 9, x_2 = 9, x_3 = 2, ...$
  ($I_3$)\\ $x_0 = 0, x_1 = 9, x_2 = 9, x_3 = 2, ...$ ($I_4: x_1 =
  x_2$)\\ $x_0 = 2, x_1 = 9, x_2 = 9, x_3 = 2, ...$ ($I_5$)\\
\end{mylisting}	
\end{esempio}

\begin{nota}		
Al momento non poniamo la nostra concentrazione su quale funzione
calcola questo programma, ma ci limitiamo ad illustrare in quale modo
opera un programma in senso puramente meccanico.
\end{nota}

\begin{extra}
Dare un programma che memorizza in $x_0$ 1 se $x_1 > x_2$ e 0
altrimenti.
\end{extra}	

Per comprendere il significato del programma e l'andamento della
computazione \`e spesso conveniente desciverlo in modo informale
attraveso un \emph{Flow Chart}, per esempio il flow chart
rappresentante il programma dell'Esempio 1.1 \`e dato in Figura
1. Notare che i test contenuti nei rombi rappresentano le istruzioni
\emph{if then else} (4) che hanno quindi due prosecuzioni in base al
risultato del test; mentre i rettangoli sono le istruzioni di
successore o assegnazione che continuano sempre con la prossima
istruzione.
		
\begin{figure}[h] 
\begin{tikzpicture}[node distance = 1.6 cm, auto]
 % Place nodes
   \node[noblock] (start) {START};
    \node [decision, below of=start] (I1) {$x_1 > x_2$};
    \node [block, below of=I1] (I2) {$x_2 \leftarrow x_2 + 1$};
    \node [block, below of=I2] (I3) {$x_3 \leftarrow x_3 + 1$};
    \node [decision, below of=I3] (I4) {$x_1 \neq x_2$};
    \node [block, below of=I4] (I5) {$x_0 \leftarrow x_3$};
   \node[noblock,below of=I5] (stop) {STOP};

    
    % Draw edges 
   \path [line] (start) -- (I1); \path [line] (I1) --
   node{yes}(I2); \path [line] (I1.east) -- ++(1.0,0) -- ++(0,-1) |-
   node [near start] {no}(I5.east); \path [line] (I2) -- (I3); \path
   [line] (I3) -- (I4); \path [line] (I4.west) -- ++(-.8,0) --
   ++(-.8,0) |- node [near start] {yes}(I2.west); \path [line] (I4)
   --node{no} (I5); \path [line] (I5) -- (stop);
\end{tikzpicture}
\caption{Flow Chart}   
	
\end{figure}
	
Dall'esempio 1 notiamo inoltre come il comando \emph{goto} nella
definizione dell'\emph{if} renda possibile l'esecuzione di cicli,
infatti l'istruzione $I_4$ riporta l'esecuzione all'istruzione $I_2$
tante volte fino a quando la condizione $x_1 \neq x_2$ non viene
soddisfatta.
	
Pi\`u in generale possiamo dire che il comando \emph{goto} pu\`o
essere utilizzato per l'esecuzione di un ciclo \emph{for}.  Infatti
con $n$ fissato il ciclo:\\
$$ \left[
\begin{array}{l}
for \; $i = 1$ \; to \; $n$\\ \ \ \ \ \left[p\;
programma\right. \\ next \; $i$\\
\end{array} \right.
\
$$ 
viene implementato dal programma nell'Esempio 1.2.

\newpage
\begin{esempio}
\end{esempio}
\begin{minipage}[c]{.50\textwidth}
\begin{mylisting}
 $I_1: x_i \leftarrow 0 $\\ $I_2: x_i \leftarrow x_i + 1 $ \\ $I_3:
  \left[p\; programma] \right. $\\ $I_4:$ if $n-x_i\neq 0 $ then \\
\hspace{\stretch{1}} goto $I_2$ else goto stop\\
\end{mylisting}
\end{minipage}
\hspace{5mm}%
\begin{minipage}[c]{.50\textwidth}
\begin{tikzpicture}[node distance = 1.6 cm, auto]
    % Place nodes 
\node[noblock] (start) {START}; \node [block, below
  of=start] (I1) {$x_i \leftarrow 0$}; \node [block, below of=I1] (I2)
{$x_i \leftarrow x_i+ 1$}; \node [block, below of=I2] (I3)
{programma}; \node [decision, below of=I3] (I4) {$n - x_i \neq 0$};
\node[noblock,below of=I4] (stop) {STOP};

    % Draw edges
\path [line] (start) -- (I1);
\path [line] (I1) -- (I2);
\path [line] (I2) -- (I3);
\path [line] (I3) -- (I4);
\path [line] (I4.west) -- ++(-.8,0) -- ++(-.8,0) |- node [near start] {yes}(I2.west);
\path [line] (I4) --node{no} (stop);

\end{tikzpicture}
\end{minipage}
\vspace{5mm}% 
	
Naturalmente, ci possono essere programmi che non terminano mai; per
esempio il semplice programma
	
\begin{mylisting}
$I_1$: if $x_i = x_i $ then goto $I_1$ else goto stop
\end{mylisting}	
non termina mai, e questo deriva dal fatto che la condizione $x_i =
x_i $ \`e sempre vera, quindi la computazione viene rimandata a $I_1$
all'infinito.
	
Ci sono molti modi pi\`u complessi per i quali una computazione
potrebbe eseguire per sempre, ma essenzialmente la causa \`e sempre
una ripetizione o un ciclo come nell'esempio citato sopra.
	
Il problema di decidere se una particolare computazione termina o meno
verr\`a trattato successivamente.\\

\subsection{Funzioni computabili da programmi}
Supponiamo che $f$ sia una funzione da $\mathbb{N}^n$ a $ \mathbb{N}$
con $(n > 1)$; cosa significa dire che $f$ \`e \emph{computabile}
da un programma? \`E naturale pensare in termini di calcolo del valore
$f(a_1, a_2, ... , a_n)$ per mezzo di un programma $P$ con
configurazione iniziale $a_1, a_2, ... , a_n$, Cio\`e, consideriamo
computazione della forma $P(a_1, a_2, ... , a_n)$.

Se tale calcolo termina, c'é un singolo numero che consideriamo come
output o risultato della computazione. Per convenzione poniamo che il
risultato alla fine della computazione sia contenuto nella variabile
$x_0$, mentre il contenuto delle altre variabili pu\`o essere
ingnorato una volta ottenuto il risultato. Diamo dunque la seguente
definizione.

\begin{programmi}
Una funzione $f$ si dice \emph{computabile} da un programma se,
assegnati i valori $a_1,...,a_n$ alle variabili $x_1,...,x_n$, il
programma restituisce il valore $x_0=f(a_1,...,a_n)$.
\end{programmi}

E' facile verificare che con una macchina a registri si possono
eseguire le stesse operazioni viste all'inizio della sezione (e
viceversa), e in effetti vale il seguente teorema.
\begin{progabaco}
Ogni funzione \'e computabile con un programma se e solo se \'e
computabile con una macchina a registri.
\end{progabaco}
\textsc{Dimostrazione} $(\Rightarrow)$ Vediamo come le macchine a
registri sono in grado di calcolare le quattro istruzioni che
costituiscono un programma:\\

\begin{enumerate}
\item 
 $x_j \leftarrow 0$ \hspace{10mm} \xymatrix{ \roundentry{j-}
  \ar@(ul,ul)[] \ar@(ul,dl)[] \ar[r] & stop }
 \vspace{10mm}

\item $x_j \leftarrow x_i $ \hspace{10mm} \xymatrix{ \roundentry{i-}
  \ar@(ul,ul)[] \ar[d] \ar[r] & \roundentry{42-} \ar@(dr,d)[d] \ar[r]
  &stop \\ \roundentry{j+} \ar[d] \ar[u] & \roundentry{i+}
  \ar[u]\\ \roundentry{42+} \ar@(dl,dl)[uu] \\ }
\vspace{10mm}

\item $x_j \leftarrow x_j+1$ \hspace{10mm} \xymatrix{ \roundentry{j+}
  \ar@(ul,ul)[] \ar[d] \\ ...  }
\vspace{10mm}

\item $\textrm{if }x_i \neq0$ then goto $\alpha$ else goto
  $\beta$ \hspace{10mm} \xymatrix{ \roundentry{i-} \ar@(ul,ul)[]
  \ar[d] \ar[r] & \beta\\ \roundentry{i+} \ar[d] \\ \alpha }

\end{enumerate}
	
$(\Leftarrow)$ Si tratta di simulare con (1)$\rightarrow$(4) le
operazioni di una macchina a registri:
\begin{enumerate}
\item esegue l'incremento di uno, la prima operazione elementare di
  una macchina a registri;\\ \\
\hspace{10mm} \xymatrix{\roundentry{j+} \ar@(ul,ul)[] \ar[d] \\ ...}
\hspace{10mm} $x_j \leftarrow x_j+1$

\item la seconda operazione elementare sulle macchine a registri si
  ottiene usando (4) e ponendo in $\alpha$ un'istruzione che sottragga
  1 al registro se non \'e vuoto, altrimenti il programma esegue
  l'istruzione $else$ $\beta$, che corrisponde all'operazione di
  procedere a destra.\\ \\ \xymatrix{ \roundentry{i-} \ar@(ul,ul)[]
    \ar[d] \ar[r] & \beta\\ \alpha }
 \hspace{10mm} $\textrm{if }x_i \neq 0$ then goto $\alpha$ else goto
 $\beta$ \\ con $\alpha = x \stackrel{\centerdot}{-} 1$ e $\beta=$
 'procedi a destra'.
\end{enumerate}

\hspace{\stretch{1}} $\Box$\\
\begin{esempio}[Differenza]
\ Come esempio riportiamo il calcolo della funzione differenza
definita come segue:\\
\begin{center}
$n \stackrel{\centerdot}{-} m= \left\{ \begin{array}{ll} n-m &
    \textrm{se } m \leq n\\ 0 & \textrm{altrimenti}
\end{array} \right.$
\end{center} 

\begin{figure}[h]
\hspace{0cm} \xymatrix{ \roundentry{1-} \ar[r] \ar@(dr,d)[d] & stop
  \\ \roundentry{2-} \ar@(ul,ul)[] \ar[u] \ar[r] & \roundentry{1-}
  \ar[d] \ar[r] & stop \\ & \roundentry{42+} \ar@(d,dl)[u] }
\caption{Macchina a registri che calcola la differenza definita sui
  naturali $n \stackrel{\centerdot}{-} m$}
\end{figure}
	
\ \\ Tale funzione viene calcolata dalla macchina a registri in Figura
2 e dal programma o che abbiamo esaminato nell'Esempio 1.1.
\end{esempio}
	
	
Il teorema appena enunciato stabilisce la completa equivalenza tra i
concetti di calcolabilità con registri e calcolabilità con
programmi. Si noti come attraverso queste equivalenze si giunga ad un
livello di astrazione sempre maggiore, e come proprio questa
progressione ci fornisca strumenti via via pi\'u versatili attraverso
i passaggi per la dimostrazione dei teoremi di incompletezza.
				

\section{Funzioni primitive ricorsive}
Vogliamo definire ora una classe che comprenda tutte e sole le
funzioni calcolabili. Seguiremo a tale scopo Post, il quale si serve
di una definizione ricorsiva. Una definizione \`e detta
\emph{ricorsiva} quando ci\`o che \`e da definirsi viene definito
facendo ricorso a istanze pi\`u elementari dello stesso problema.
Tale metodo consiste nel:
\begin{itemize}
 \item fissare un insieme di funzioni iniziali immediatamente
   calcolabili quale base della procedura di definizione
 \item indicare alcune regole per derivare altre funzioni ricorsive da
   quelle date in partenza (regole che ovviamente preservino la
   calcolabilità delle funzioni derivate)
 \item escludere dalla classe delle funzioni ricorsive quelle funzioni
   che non siano le iniziali o da queste derivabili.
\end{itemize}
Diamo dunque la seguente definizione:

\begin{programmi}
Si dice \emph{funzione primitiva ricorsiva} un elemento della
classe definita induttivamente a partire dalle seguenti funzioni:
\begin{itemize}
\item [a.] la funzione nulla $z$ tale che $z(x)=0$;
\item [b.] la funzione proiezione $p^n_i$ tale che $p_i^n(x_1,\ldots
  ,x_n)=x_i$ $\forall i\leq n$
\item [c.]la funzione successore $s$ tale che $s(x) = x+1$;
\end{itemize}
\end{programmi}
	
Le regole per produrre nuove funzioni sono:
\begin{enumerate}
\item le operazioni di \emph{\emph{composizione}}, che date le
  funzioni primitive ricorsive $f: \mathbb{N}^k \rightarrow
  \mathbb{N}$ e $g_i : \mathbb{N}^n \rightarrow \mathbb{N}$ per $i= 1,
  ... , k$ permette di ottenere una funzione $h : \mathbb{N}^n
  \rightarrow \mathbb{N}$ per cui:\\ $
  h(x_1,.....,x_n)=f(g1(x_1,.....,x_n),....., g_k (x_1,.....,x_n))$
  \\ anch'essa primitiva ricorsiva.
\item lo schema di \emph{\emph{ricorsione primitiva}}, che date
  le funzioni primitive ricorsive $f: \mathbb{N}^k \rightarrow
  \mathbb{N}$ e $g : \mathbb{N}^{k+2} \rightarrow \mathbb{N}$, allora
  \`e primitiva ricorsiva anche la funzione $h : \mathbb{N}^{k+1}
  \rightarrow \mathbb{N}$ che soddisfi il sistema di equazioni:
\begin{center}
$\left\{ \begin{array}{ll} h(x_1,.....,x_n,0)= f(x_1,.....,x_n)
    \\ h(x_1,.....,x_n,s(y))=g(x_1,.....,x_n, y, h(x_1,.....,x_n,y))
				\end{array} \right.$
\end{center}
\end{enumerate}
\vspace{5mm}%
	
La classe delle funzioni primitive ricorsive \`e dunque \emph{chiusa
  rispetto alle operazioni di composizione e ricorsione
  primitiva}. Ora vedremo che questa classe di funzioni, che abbiamo
appena definito in termini matematici, \`e calcolabile dai programmi.

% Ane Santos Herranz

\begin{thm}
\emph{Ogni funzione primitiva ricorsiva è eseguibile da un programma
(e quindi calcolabile).}\end{thm}
\begin{proof}
Dobbiamo fare un programma per le tre funzioni di base, per la
composizione generalizzata e la ricorsione. Come sappiamo il output è
sempre in $x_{0}$
\begin{itemize}
\item[a.] Facciamo un programma per la funzione zero:
\begin{mylisting}
$I_1$: $x_{0}\leftarrow0$
\end{mylisting}

\item[b.] Il programma per il successore:\begin{mylisting}$I_1$:
  $x_{0}\leftarrow x_{1}+1$\end{mylisting}

\item[c.] La proiezione:\begin{mylisting}$I_1$: $x_{0}\longleftarrow
  x_{i}$\end{mylisting}
\end{itemize}

\begin{enumerate}
\item La composizione: siano $P, P_1, P_2, \dots, P_n$ programmi per
  le funzioni $f,g_{1},...g_{n}$ rispettivamente. L'output di ogni
  funzione viene salvato in $x_{0}$; quindi, poich\'e vogliamo salvare
  tutti i risultati intermedi del calcolo di $g_{1},\dots,g_{n}$
  dobbiamo copiare di volta in volta $x_0$ in una variabile che siamo
  sicuri che non sia usata da uno dei programmi. Inoltre dobbiamo
  riservarci dello spazio per le variabili di input poich\'e qualche
  programma potrebbe modificarle durante la sua esecuzione.  Per
  sapere dove possiamo salvare tutti questi valori dobbiamo sapere
  quali variabili vengono utilizzate dai programmi.
  Sia quindi
  $$\rho(Q)=max\left\{ i|x_{i}\leftarrow...\in Q\right\}$$
  il massimo indice di una variabile assegnata in $Q$ e
  $$\varepsilon=max\left\{
  \rho(P_{1}),....,\rho(P_{n}),\rho(P),k\right\}$$
  il massimo indice utilizzato nei programmi $P, P_1, \dots, P_n$.

Costruiamo il programma che calcola la composizione come segue:
salviamo l'input, che si trova in $x_{1},\dots,x_{k}$ in
$x_{\varepsilon+1},\dots,x_{\varepsilon+k}$, applichiamo il programma
$P_1$ e copiamo l'output in $x_{\varepsilon+k+1}$:

\begin{mylisting}
$x_{\varepsilon+1}\leftarrow x_{1}$\\
$\vdots$\\
$x_{\varepsilon+k}\leftarrow x_{k}$\\
$P_{1}$\\
$x_{\varepsilon+k+1}\leftarrow x_{0}$
\end{mylisting}

Copiamo l'input:
\begin{mylisting}
$x_{1}\leftarrow x_{\varepsilon+1}$\\
$\vdots$\\
$x_{k}\leftarrow x_{\varepsilon+k}$
\end{mylisting}

Applichiamo $P_{2}$ e copiamo l'output in $x_{\varepsilon+k+2}$;
ripetiamo questa operazione per tutti gli altri programmi fino ad
applicare il programma $P_{n}$ e copiare il suo output in
$x_{\varepsilon+k+n}$.

A questo punto tutti gli input per P si trovano in
$x_{\varepsilon+k+1},\dots,x_{\varepsilon+k+n}$.  Copiamo questi
valori in $x_{1},\dots,x_{k}$ e applichiamo il programma P:

\begin{mylisting}
$x_{1}\leftarrow x_{\varepsilon+k+1}$\\
$\vdots$\\
$x_{k}\leftarrow x_{\varepsilon+k+n}$\\
$P$
\end{mylisting}

Dopo questa sequenza di operazioni il risultato della composizione si
trova in $x_{0}$.

\item La ricorsione: sia $F$ un programma che calcola $f$ e $G$ un
  programma per $g$. Per realizzare un programma che calcola $h$
  dobbiamo innanzitutto salvare le variabili di input e i risultati
  parziali di $f$ e $g$ in variabili non utilizzate dai due programmi
  $F$ e $G$.

  Sia $\rho(Q)=max\left\{ i|x_{i}\leftarrow\dots \in Q\right\}$ il
  massimo indice utilizzato nel programma Q; definiamo
  $\varepsilon=max\left\{ \rho(F),\rho(G),k+2\right\}$ il massimo
  indice utilizzato per entrambi i programmi.

Allora, prima copiamo l'input e mettiamo 0 in $x_{\varepsilon+k+2}$
per cominziare la ricorsione (per calcolare $h(x,0)$). Adesso possiamo
applicare il programma $F$.

Poi dobbiamo guardare se l'$y$ che ci è stato dato all'inizio è uguale
a zero; se è cosi, il programma deve terminare. Se non è cosi, il
programma deve calcolare $h(x,s(y))$ (la prima volta $s(y)$ sarà
1). Il programma G utilizza come input $x_{1},\dots,x_{k}$ per
$x_{1},\dots,x_{k}$, $x_{k+1}$ per $y$ e $x_{k+2}$ per il risultato
parziale $h(x,y)$. Quindi dobbiamo mettere l'input iniziale in
$x_{1},\dots,x_{k}$, il risultato al passo precedente che si trova in
$x_{\varepsilon+k+2}$ dobbiamo copiarlo in $x_{k+1}$ e, dopo aver
eseguito $G$, dobbiamo copiare il suo output da $x_{0}$ in $x_{k+2}$.
Infine calcoliamo il successore di $y$ e controlliamo se equivale
all'$y$ iniziale: in caso affermativo dobbiamo terminare, altrimenti
torniamo ad eseguire il ciclo. Allora, il programa sarà:

\begin{mylisting}
$x_{\varepsilon+1}\leftarrow x_{1}$\\
$\vdots$\\
$x_{\varepsilon+k}\leftarrow x_{k}$\\
$x_{\varepsilon+k+1}\leftarrow x_{k+3}$\\
$F$\\
$x_{\varepsilon+k+2}\leftarrow x_{0}$\\
$x_{\varepsilon+k+1}\leftarrow 0$\\
$loop:\; $if$\; x_{\varepsilon+k+1}=x_{\varepsilon+k+3}\; $then goto$\;stop$\\
$x_{\varepsilon+k+1}\leftarrow x_{\varepsilon+k+1}+1$\\
$x_{1}\leftarrow x_{\varepsilon+1}$\\
$\vdots$\\
$x_{k}\leftarrow x_{\varepsilon+k}$\\
$x_{k+1}\leftarrow x_{\varepsilon+k+1}$\\
$x_{k+2} \leftarrow x_{\varepsilon+k+2}$\\
$G$\\
$x_{\varepsilon+k+2} \leftarrow x_0$\\
goto$\; loop$\\
$stop:$
\end{mylisting}
\end{enumerate}
\end{proof}

% Controllare
\subsection{Esempi di funzioni primitive ricorsive}

\begin{esempio}[somma]
$h:\mathbb{N}^2 \to \mathbb{N}$, $h(x,y)=x+y$.
Possiamo usare lo schema semplificato
$\left\{ \begin{array}{ll}
	x+0=x\\
	x+s(y)=s(x+y)
\end{array}\right.$.\\
Nel nostro schema di ricorsione la stessa funzione si ottiene con $h$
tale:\newline
$$\begin{array}{ll}
	h(x,0)=f(x)=p_1^1(x)=x\\
	h(x,s(y))=g(x,y,h(x,y))=s(p_3^3(x,y,h(x,y)))=s(h(x,y)).
\end{array}$$\newline
\end{esempio}
%
\begin{esempio}[prodotto]

 $h:\mathbb{N}^2 \to \mathbb{N}$, $h(x,y)=x \cdot y$ che possiamo scrivere
come
$\left\{ \begin{array}{ll}
	x \cdot 0=0\\
	x \cdot s(y)= x \cdot y + x
\end{array}\right.$ e quindi: \newline
$$\begin{array}{ll}
	h(x,0)=f(x)=z(x)\\
	h(x,s(y))=g(x,y,h(x,y))=p_3^1(x,y,h(x,y))+p_3^3(x,y,h(x,y)).
\end{array}$$\newline
\end{esempio}
%
\begin{esempio}[fattoriale] Per adattare la definizione della funzione
fattoriale
$h:\mathbb{N} \to \mathbb{N}$, $h(x)=x!$ allo schema generale si considera
$\left\{ \begin{array}{ll}
	0!=1\\
	s(y)! = y! \cdot s(y)
\end{array}\right.$ e quindi si pu\`o prendere $h$ come segue: \newline
$$\begin{array}{ll}
	h(x,0)=f(x)=s(z(x))=1\\
	h(x,s(y))=g(x,y,h(x,y))= s(p_3^2(x,y,h(x,y)))) \cdot p_3^3(x,y,h(x,y))=
\\
	\hspace{1.8cm} = s(y)\cdot h(x,y).
	\end{array}$$\newline
\end{esempio}
%
%
\begin{esempio}[elevamento a potenza] $h:\mathbb{N}^2 \to \mathbb{N}$,
$h(x,y)=x^y$:
$\left\{ \begin{array}{ll}
	x^0=1\\
	x^{s(y)} = x^y \cdot x
\end{array}\right.$ e quindi: \newline
$$\begin{array}{ll}
	h(x,0)=f(x)=s(z(x))=1\\
	h(x,s(y))=g(x,y,h(x,y))= p_3^1(x,y,h(x,y)) \cdot p_3^3(x,y,h(x,y)) =
x\cdot h(x,y).
	\end{array}$$\newline
\end{esempio}
%
%
\begin{esempio}[predecessore] $p:\mathbb{N} \to \mathbb{N}$ tale che
$\left\{ \begin{array}{ll}
	p(0)=0\\
	p(s(y)) = y
\end{array}\right.$ e quindi: \newline
$$\begin{array}{ll}
	p(0)=f=0\\
	p(s(y))=g(y,p(y))= p_2^1(y,p(y)).
	\end{array}$$\newline
\end{esempio}
%
%
\begin{esempio}[sottrazione] $h:\mathbb{N}^2 \to \mathbb{N}$,
$ h(x,y) = \left\{ \begin{array}{ll}
	x \stackrel{\centerdot}{-} y \ se \ y \leq x\\
	0 \ altrimenti
\end{array}\right.$ dunque: \newline
$$\begin{array}{ll}
	h(x,0)=f(x)=x\\
	h(x,s(y))=p(h(x,y)).
	\end{array}$$\newline
\end{esempio}
%
%
\begin{esempio}[segno] $sgn:\mathbb{N} \to \left\{0,1\right\}$,
$ sgn(y) = \left\{ \begin{array}{ll}
	0 \ se \ y = 0\\
	1 \ se \ y > 0
\end{array}\right.$ che si pu\`o scrivere come:
 \[ sgn(y) = y \stackrel{\centerdot}{-} p(y) \]
che \`e primitiva ricorsiva per quanto visto negli esempi precedenti.
\end{esempio}
%
%
\begin{esempio}[controsegno] $\overline{sgn}:\mathbb{N} \to
\left\{0,1\right\}$,
$ \overline{sgn}(y) = \left\{ \begin{array}{ll}
	1 \ se \ y = 0\\
	0 \ se \ y > 0
\end{array}\right.$ che si pu\`o pensare come:
 \[ \overline{sgn}(y) = 1 \stackrel{\centerdot}{-} sgn(y) \]
dunque \`e primitiva ricorsiva.
\end{esempio}
%
%
\begin{esempio}[] $f(\overrightarrow{x},y)= \sum_{i=0}^{y} g(\overrightarrow{x},
i)$ con $g$ primitiva ricorsiva, allora:\\
 $ f(\overrightarrow{x}, 0) = g(\overrightarrow{x}, 0)  $ \\
 $ f(\overrightarrow{x}, s(y)) = f(\overrightarrow{x}, y) +
g(\overrightarrow{x}, s(y)). $
\end{esempio}
%
%
\begin{esempio}[] $f(\overrightarrow{x},y)= \prod_{i=0}^{y}
g(\overrightarrow{x}, i)$ con $g$ primitiva ricorsiva, allora:\\
$	f(\overrightarrow{x}, 0) = g(\overrightarrow{x}, 0) $ \\
$ 	f(\overrightarrow{x}, s(y)) = f(\overrightarrow{x}, y) \cdot
g(\overrightarrow{x}, s(y)). $
\end{esempio}
%
%
\begin{esempio}[] $ \chi_{\geq}(x,y) = \left\{ \begin{array}{ll}
	1 \ se \ x \geq y\\
	0 \ altrimenti
\end{array}\right.$
\[ \chi_{\geq}(x,y) = sgn (s(x) \stackrel{\centerdot}{-} y) . \]
\end{esempio}
%
%
\begin{esempio}[valore assoluto] Basta scriverlo come
\[ \left|x - y \right| = (x \stackrel{\centerdot}{-} y) + ( y
\stackrel{\centerdot}{-} x) \]
oppure
\[ \left|x - y \right| = (x \stackrel{\centerdot}{-} y) \chi_{\geq}(x,y)  +
( y \stackrel{\centerdot}{-} x) (1 - \chi_{\geq}(x,y)) .\]
\end{esempio}
%
%
\begin{esempio}[] $f(\overrightarrow{x}) = \left\{ \begin{array}{ll}
	g_1(\overrightarrow{x})\ se\ vale\  R(\overrightarrow{x})\\
	g_2(\overrightarrow{x})\ altrimenti
\end{array}\right.$ con \\$\chi_R(\overrightarrow{x}) = \left\{
\begin{array}{ll}
	1\ se\ R(\overrightarrow{x})\ vale\\
	0\ altrimenti
\end{array}\right.$ con $g_1(\overrightarrow{x})$, $g_2(\overrightarrow{x})$
e $\chi_R(\overrightarrow{x})$ primitive ricorsive.\\
Allora
\[ f(\overrightarrow{x}) = g_1(\overrightarrow{x}) \chi_R(\overrightarrow{x}) +
 g_2(\overrightarrow{x}) (1 - \chi_R(\overrightarrow{x})). \]
\end{esempio}
%

% Alessandro Onnivello
\section{Dalle funzioni primitive ricorsive alle funzioni ricorsive}

Nel paragrafo precedente abbiamo definito l'insieme delle funzioni
primitive ricorsive; possiamo dire che queste siano sufficienti per
rappresentare tutte le funzioni calcolabili da una macchina di Turing?
La risposta è negativa, come andremo ora a dimostrare.

\begin{prop}\label{PRsonoTotali}
Le funzioni primitive ricorsive sono totali.
\end{prop}
\begin{proof}
Si dimostra per induzione sulla struttura delle funzioni primitive
ricorsive.  Poichè le funzioni di base sono funzioni totali dobbiamo
verificare che la composizione e la ricorsione primitiva preservano la
totalità delle funzioni composte.

Per la composizione dobbiamo verificare che la funzione composta
$f(g_1, \dots, g_k)$ è totale. Per ipotesi induttiva
(strutturale) $f,g_1,\dots,g_k$ sono totali. Per la totalità di
$g_1,\dots,g_k$ tutti gli argomenti di $f$ sono definiti e, poichè $f$
è totale, anche la composizione è definita.

Per la ricorsione primitiva dobbiamo dimostrare che
$h:\mathbb{N}^{k+1} \to \mathbb{N}$ è definita $\forall y \in
\mathbb{N}$ e $\forall \vec{x} \in \mathbb{N}^k$.

$$h(\vec{x},y)=\left\{
\begin{array}{ll} h(\vec{x}, 0) = f(\vec{x})\\
                  h(\vec{x}, s(y)) = g(\vec{x},y,h(\vec{x},y))
\end{array} \right.$$

Si dimostra per induzione su $y$.

$y = 0$: $h(\vec{x}, 0) = f(\vec{x})$\\ $f$ è totale per ipotesi
induttiva quindi $\forall \vec{x} \in \mathbb{N}^k (\vec{x}, 0) \in
dom(h)$ come atteso

$y + 1$: $h(\vec{x}, y + 1) = g(\vec{x}, y, h(\vec{x}, y))$\\
$h(\vec{x}, y)$ è totale per ipotesi induttiva su $y$ quindi
$h(\vec{x}, y)$ è definita. Poichè $g$ totale per ipotesi induttiva e
$h(\vec{x}, y)$ è definita allora $\forall \vec{x} \in \mathbb{N}^k \;
(\vec{x}, y + 1) \in dom(h)$.
\end{proof}

Ora, per quanto detto nella precedente proposizione, l'insieme delle funzioni
primitive ricorsive permette di rappresentare solo funzioni calcolabili totali
ma, come abbiamo visto nel primo capitolo, le macchine di Turing possono
calcolare anche funzioni parziali. Quindi possiamo affermare che le
funzioni primitive ricorsive non rappresentano tutte le funzioni Turing
calcolabili.

Ma limitandoci al caso delle funzioni totali, possiamo dire che queste
siano tutte primitive ricorsive? Anche in questo caso la risposta è
negativa come vedremo nel prossimo paragrafo.

\section{Metodo diagonale di Cantor}
Per poter dimostrare che l'insieme delle funzioni primitive ricorsive
non contiene tutte le funzioni totali calcolabili andremo ora a
introdurre il metodo diagonale (o diagonalizzazione) di Cantor.  Il
metodo di diagonalizzazione è una tecnica dimostrativa, ideata da
Georg Cantor per provare la non numerabilità dei numeri reali, che
risulta molto utile anche nell'ambito della logica matematica e della
teoria della computabilità come vedremo in seguito.

Il metodo diagonale di Cantor consiste nel costruire una funzione $\chi$
che differisce da un'insieme infinito di funzioni $\chi_0,\chi_1,\dots$
effettivamente enumerabile. Si costruisce quindi la seguente tabella
dove ogni colonna contiene una funzione unaria e le righe contengono
la sequenza dei naturali, ovvero tutti i possibili argomenti.
\begin{table}[!h]
\begin{center}
\begin{tabular}{|c|c|c|c|c}

\hline
 & $\chi_0$ & $\chi_1$ & $\chi_2$ & $\ldots$\\
\hline
0 & $\chi_0(0)$ & $\chi_1(0)$  & $\chi_2(0)$  & $\ldots$\\
\hline
1 & $\chi_0(1)$ & $\chi_1(1)$  & $\chi_2(1)$  & $\ldots$\\
\hline
2 &  $\chi_0(2)$ & $\chi_1(2)$  & $\chi_2(2)$  & $\ldots$\\
\hline
$\vdots$ & $\vdots$ &       $\vdots$  & $\vdots$  & $\ddots$\\

\end{tabular}
\end{center}
\caption{Metodo diagonale di Cantor}
\end{table}

A questo punto possiamo costruire una funzione $\chi$ in modo tale che
differisca da ogni funzione enumerata nella tabella; vogliamo quindi
costruire una funzione tale che $\forall i \; \chi(i) \neq
\chi_{i}(i)$, ovvero che differisce da ogni altra funzione elencata
nella tabella almeno sulla diagonale. Questa nuova funzione, creata
tramite una procedura effettiva, non può appartenere all'insieme di
partenza, poichè altrimenti differirebbe da sè stessa sulla diagonale.

\begin{prop}\label{TotCalcNonNum}
Data una qualunque lista \emph{effettiva} di tutte le funzioni totali
calcolabili, unarie per semplicit\`a, $f_{0},\; f_{1},\; f_{2},
\cdots$ esiste sempre una funzione $g$ totale calcolabile che non
compare nella lista.
\end{prop}
\begin{proof}
Sia $f_{0},\; f_{1},\; f_{2}, \cdots$ una lista effettiva di funzioni
totali calcolabili, allora costruiamo la funzione $g$ nel seguente
modo:

\begin{center}
$g: \mathbb{N} \to \mathbb{N}$\\
$g(x) = f_{x}(x) + 1$
\end{center}
\begin{table}[!h]
\begin{center}
\begin{tabular}{|c|c|c|c|c}

\hline
 & $f_0$ & $f_1$ & $f_2$ & $\ldots$\\
\hline
0 & $f_0(0) \mathbf{+1}$ & $f_1(0)$  & $f_2(0)$  & $\ldots$\\
\hline
1 & $f_0(1)$ & $f_1(1) \mathbf{+1}$  & $f_2(1)$  & $\ldots$\\
\hline
2 &  $f_0(2)$ & $f_1(2)$  & $f_2(2) \mathbf{+1}$  & $\ldots$\\
\hline
$\vdots$ & $\vdots$ &       $\vdots$  & $\vdots$  & $\ddots$\\
\end{tabular}
\end{center}
\caption{Funzione di diagonalizzazione $\mathbf{g}$}
\label{diagG}
\end{table}

$g$ è sicuramente totale calcolabile poichè, essendo la lista di
funzioni effettiva è quindi possibile, dato un qualsiasi $n \in
\mathbb{N}$, calcolare l'ennesima funzione, valutarla in $n$ ed
aggiungerci 1. È chiaro che $g$ non può appartenere alla lista, poichè
presa una qualsiasi colonna $n$ della tabella~\ref{diagG} che
rappresenta l'immagine dell'ennesima funzione, alla riga ennesima
$g(n) \neq f_{n}(n)$ poichè per definizione $g(n)=f_{n}(n)+1$. Perciò
non esiste alcun $n$ tale che $f_{n}(x)=g(x)$ per ogni $x \in
\mathbb{N}$ e quindi $g$ non è contenuta nella nostra lista.
\end{proof}

Fatto questo ragionamento, ne consegue che non è possibile enumerare
in modo effettivo tutte le funzioni totali calcolabili.

Possiamo usare ora la proposizone~\ref{TotCalcNonNum} per dimostrare il seguente

\begin{thm}\label{diagRic} Esiste una funzione totale calcolabile che non
è primitiva ricorsiva.
\end{thm}

\begin{proof}
Dobbiamo innanzitutto costruire una lista \emph{effettiva} di tutte le
funzioni primitive ricorsive (per semplicit\`a, supponiamole ad un
argomento).

Cerchiamo di scrivere informalmente un algoritmo che ci permetta di
listare tutte queste funzioni (in seguito vedremo come si pu\`o fare
in modo rigoroso).  Una tale enumerazione si pu\`o ottenere nello
stesso modo in cui si ottiene una lista di teoremi a partire dagli
assiomi di una teoria. Si prende prima una funzione base, poi si
``contano'' tutte le funzioni ottenute da questa mediante una sola
applicazione di composizione e/o ricorsione. Poi si prende la seconda
funzione e tutte le funzioni derivate mediante una applicazione delle
due regole da quest'ultima e/o dalle funzioni gi\`a ottenute al passo
precedente, e cos\`i via. In questa maniera non si tralascia alcuna
funzione e ad ogni passo la lista \`e finita. Abbiamo cos\`i ottenuto
una enumerazione effettiva per tutte le funzioni dell'insieme
$\mathcal{PR}$.\\ Ma per quanto detto nella
proposizone~\ref{TotCalcNonNum} sappiamo che le funzioni totali non
sono enumerabili effettivamente quindi questo implica che esiste una
funzione totale calcolabile che non appartiene l'enumerazione che
abbiamo appena definito e che quindi non appartiene all'insieme
$\mathcal{PR}$.
\end{proof}

A questo punto ci chiediamo: cosa ci manca per avere tutte le funzioni
calcolabili?
Ripensando a quanto visto fino ad ora, il problema pu\`o stare o nel pretendere
di avere una lista effettiva delle funzioni calcolabili (cosa che per\`o \`e
ragionevole a farsi: lo abbiamo visto poco fa), o nell'imporre la propriet\`a di
totalit\`a alle funzioni. Dunque vediamo se lasciando cadere questa assunzione
riusciamo a classificare tutte le funzioni calcolabili.


\section{Le funzioni ricorsive}
Avendo osservato che le funzioni primitive ricorsive non
esauriscono tutte le funzioni calcolabili, il passo successivo \`e  quello di
estenderle ulteriormente. In particolare ci sar\`a bisogno di trovare
anche funzioni parziali nella nostra definizione di ``ricorsivit\`a''. Perci\'o
introduciamo
la seguente funzione: \\

f. \textbf{minimizzazione}: sia $f: \mathbb{N}^{d+1} \to \mathbb{N}$ (anche
parziale)\\
definiamo la funzione di minimizzazione $h: \mathbb{N}^{d} \to \mathbb{N}$ come
$$ h(\vec{x}):= \mu_{y}(f(\vec{x},y)) $$
dove

\[\mu_{y}(f(\vec{x},y)) = 
\begin{cases}
il \; minimo \; y \; tale \; che: \\
\qquad (i) \; f(\vec{x},z)\downarrow \; \forall z \leq y \; e \\
\qquad (ii) \; f(\vec{x},y)=0 & \text{se y esiste,} \\
non \; definita \; & \text{se $\exists z < y \; f(\vec{x},z) \uparrow$}\\
 & \text{o se $f(\vec{x},y)\neq 0 \; \forall y \in \mathbb{N} $}
\end{cases} \]

con $\mu$ che prende il nome di operatore di minimizzazione.
\begin{thm} Le funzioni ricorsive sono b-programma-calcolabili.
\end{thm}
\begin{proof} Poich\`e abbiamo gi\`a dimostrato che le tre funzioni di base
(funzione zero, successore e proiezioni), la composizione generalizzata e la
ricorsione primitiva sono calcolabili da un b-programma ci resta solo da far
vedere che questo vale anche per la minimizzazione. Infatti se
$f:\mathbb{N}^{d+1} \to \mathbb{N}$ \`e computabile con un b-programma $\alpha$
allora possiamo eseguire la seguente assegnazione:
$$x_i \leftarrow f(\vec{x_{j}}, x_k)$$
(dove con $\vec{x_{j}}$ vogliamo rappresentare in modo compatto una lista di d
variabili di input)che significa "`dai come input $\vec{x_{j}}$ e $x_k$ al
b-programma $\alpha$ che calcola $f(\vec{x_{j}}, x_k)$ e dai output risultante
come valore a $x_i$"'. Dunque il b-programma che calcola la funzione di
minimizzazione su $\vec{x}$ \`e il seguente:\\
   \begin{mylisting}
       $x_0 \leftarrow 0$ \\
       $loop$: $\vec{x_{1}} \leftarrow \vec{x}$\\
       $x_{d+2} \leftarrow f(\vec{x_{1}}, x_0)$\\
       if $x_{d+2} = 0$ then goto $stop$\\
       $x_0 \leftarrow x_0 + 1$\\
       goto $loop$
  \end{mylisting}
\end{proof}

Una volta dimostrato che la minimizzazione è calcolabile dai
b-programmi possiamo dare la seguente
\begin{defi} Una funzione $f:\mathbb{N}^{d} \to \mathbb{N}$ (totale o
parziale) si dice \emph{ricorsiva} se si ottiene dalle funzioni
iniziali (a.--c.) applicando la composizione, la ricorsione e la
minimizzazione (d.--f.).
\end{defi}

Una cosa importante da notare \`e che l'utilizzo dell'operatore $\mu$ consente
di ottenere funzioni parziali anche a partire funzioni totali, come vedremo nei
seguenti esempi.
\begin{esempio} data $f:\mathbb{N} \to \mathbb{N}$ definita nel seguente modo:
$$f(x)= \left\{ \begin{array}{ll}
il \; min \; y \; t. \, c. \; x+y=0 & se \; \exists \; y\\
non \; definita & se \; x+y \neq 0 \; \forall y \in \mathbb{N}\\
\end{array} \right.$$
utilizzando la minimizzazione pu\`o essere scritta nel seguente modo:
$$f(x)= \mu_{y}(x+y)$$

Si noti che, essendo $x \in \mathbb{N}$, se $x \neq 0 \; f$ non \`e definita
anche se la funzione somma su cui viene applicato l'operatore di minimizzazione
\`e totale.
\end{esempio}

\begin{esempio} data $f:\mathbb{N}^{2} \to \mathbb{N}$ definita nel seguente
modo:
$$f(x,y)= \left\{ \begin{array}{ll}
x/y & se \; y \vert x\\
non \; definita & se \; y \nmid x\\
\end{array} \right.$$
pu\`o essere scritta nel seguente modo:
$$f(x,y) = \mu_{z}(\vert (z*y) - x \vert) $$
anche in questo caso la funzione $g(x,y,z)=\vert (z*y) - x \vert$ \`e totale, ma
combinata con l'operatore di minimizzazione consente di ottenere la funzione
parziale $f$.
\end{esempio}

\section{Relazioni ricorsive}
\begin{defi} $R \subseteq \mathbb{N}^{n}$ è una \emph{relazione
ricorsiva} se esiste una funzione ricorsiva $\chi_R$ che assume solo valori 0 e
1 e che soddisfa
$$\chi_R(x_1, \ldots, x_n):= \left \{ \begin{array}{ll}
                                      1 & (x_{1}, x_{2}, \cdots, x_{n}) \in R\\
                                      0 & (x_{1}, x_{2}, \cdots, x_{n}) \not \in
 R\\
                                      \end{array} \right. $$
\end{defi}
\begin{prop} Se $R, S \subseteq \mathbb{N}^{n}$ sono relazioni
ricorsive allora anche $R\land S$, $R \vee S$, $\neg R$ sono relazioni
ricorsive. Inoltre, se $w \in \mathbb{N}$, anche $\forall y \leq
w. R(x_1, \ldots, x_{n-1}, y)$ e $\exists y \leq w. R(x_1, \ldots,
x_{n-1}, y)$ sono relazioni ricorsive.
\end{prop}
\begin{proof} Si pone
\begin{itemize}
\item $\chi_{R\land S} = \chi_R \chi_S$;
\item $\chi_{R\vee S} = \chi_R + \chi_S - \chi_R  \chi_S$;
\item $\chi_{\neg R} = 1 - \chi_R$;
\item $\chi_{\forall y
\leq w. R(x_1, \ldots, x_{n-1}, y)} = \prod_{i=0}^{w} \chi_R(x_1, \ldots,
x_{n-1},i)$;
\item $\chi_{\exists y
\leq w. R(x_1, \ldots, x_{n-1}, y)} = sgn(\sum_{i=0}^{w} \chi_R(x_1, \ldots,
x_{n-1},i))$
\end{itemize}
\end{proof}

Grazie alla definizione data sopra, ci \`e possibile giustificare la
\emph{definizione per casi} delle funzioni ricorsive.

Supponiamo di avere una funzione $f: \mathbb{N}^{n} \to n$ t. c.
$$f(\vec{x}):= \left \{ \begin{array}{ll}
                    g_1(\vec{x}) & $se $ \vec{x} \in R_{1}\\
                    \vdots \\
                   g_k(\vec{x}) & $se $ \vec{x} \in R_{k}\\
                    \end{array} \right.$$
con $g_{1}, g_{2}, \cdots, g_{k}:\mathbb{N}^{n} \to \mathbb{N}$ funzioni
ricorsive e dove $R_{1}, R_{2}, \cdots, R_{k} \subseteq \mathbb{N}^{n}$ sono
relazioni ricorsive mutualmente esclusive ed esaustive, ossia abbiamo che:
\begin{itemize}
\item $\sum_{i=1}^{k} \chi_{R_{i}}(\vec{x})=1$ e che
\item se $R_{i}(\vec{x})$ vale allora $\forall j \neq i \; \chi_{R_{i}}(\vec{x})
\cdot \chi_{R_{j}}(\vec{x})=0$.
\end{itemize} 
Possiamo quindi prendere le funzioni caratteristiche $\chi_{R_{1}},
\chi_{R_{2}}, \cdots, \chi_{R_{k}}$ e definire la funzione $f$ come:
$$f(\vec{x})= \sum_{i=1}^{k} g_i(\vec{x})\chi_{R_{i}}(\vec{x})$$ e
quindi essendo $f$ ottenuta per composizione di funzioni ricorsive \`e
anch'essa ricorsiva.  Quindi se vogliamo descrivere una funzione che
assume valori diversi in casi diversi lo possiamo fare mantenendoci
all'interno delle funzioni primitive ricorsive. Questo può permetterci
di definire funzioni come quella del prossimo esempio.

\begin{esempio} Data la seguente funzione:
\[ f: \mathbb{N}^{2} \to \mathbb{N} \]
\[ f(x,y)= \begin{cases}
x-y	& \text{se $x \geq y$}\\
x+y & \text{se $x<y$}
\end{cases}\]
la sua definizione per casi \`e la seguente:\\
$f(x,y) = \sum_{i=1}^{2} g_{i} \cdot \chi_{R_{\geq}}(x,y) = (x-y)\cdot
\chi_{R_{\geq}}(x,y) + (x+y)\cdot \neg \chi_{R_{\geq}}(x,y) $
Si lascia come esercizio al lettore il trovare $\chi_{R_{\geq}}$.
\end{esempio}

% Alessandro Bruni

\section{Equivalenze}

Per riassumere il percorso seguito finora presentiamo il seguente schema.
Ad ogni passaggio abbiamo introdotto un linguaggio pi\`u
raffinato senza per\`o aggiungere nuove funzioni:
 ogni volta abbiamo aggiunto qualcosa che fosse definibile in
termini del precedente e quindi in particolare di macchine di Turing.
\begin{align*}
{\rm Macchine\ }&{\rm di\ Turing\ }\\
\bigcup &|\\
{\rm Macchine\ }&{\rm a\ registri\ }\\
\bigcup &|\\
{\rm Progra}&{\rm mmi}\\
\bigcup &|\\
{\rm Funzioni\ }&{\rm ricorsive\ }\\
\bigcup &| \; {\rm?}\\
{\rm Macchine\ }&{\rm di\ Turing\ }\\
\end{align*}

È valida valida l'ultima inclusione? in caso affermativo tutti i
concetti di funzioni calcolabili descritti fino ad ora sono tra loro
equivalenti. Il seguente teorema dimostra che quest'inclusione è
effettivamente valida.

\begin{thm}[Le funzioni Turing-calcolabili sono ricorsive]
Ovvero, per ogni MdT possiamo costruire una funzione ricorsiva che la calcola.
\end{thm}

\begin{proof}
Dobbiamo innanzitutto trovare un modo per rappresentare il nastro (ovvero la
memoria) della MdT. Per semplicità di esposizione poniamoci nel caso $\sigma =
\{ B, 1 \}$, poichè ogni altro alfabeto finito può avere una rappresentazione
simile nei naturali, scegliendo una rappresentazione in base $|\sigma|$
dell'input.

Una MdT computa una funzione $f: \mathbb{N}^k \rightarrow \mathbb{N}$,
possibilmente parziale, quindi lo stato del nastro all'inizio della
computazione è una sequenza di caratteri che contiene le variabili in
input:
$$\dots BB\overline{x_1+1}B\overline{x_2+1}B...B\overline{x_k+1}BB\dots $$
mentre alla fine del calcolo la macchina dovrebbe lasciare nel nastro
il risultato della funzione che calcola:
$$\dots BB\overline{f(x_1,\dots,x_k)+1}BB\dots $$

Diamo ora le seguenti definizioni, per meglio comprendere la nostra codifica
del nastro in $\mathbb{N}$.

\begin{defi}[Numerale sinistro]
Si definisce numerale sinistro la sequenza di caratteri, letti da sinistra a
destra a partire dal primo 1, che si trovano strettamente a sinistra della
testina della MdT, dove 1 rappresenta la cifra 1 e $B$ rappresenta la cifra 0.
\end{defi}
\begin{defi}[Numero sinistro]
Si dice numero sinistro l'interpretazione nei naturali del numerale sinistro,
ovvero quel numero $n \in \mathbb{N}$ la cui rappresentazione in base 2 è
esattamente il numerale sinistro.
\end{defi}
\begin{defi}[Numerale destro]
È la sequenza di caratteri a destra della testina (compreso quello sopra la
testina stessa) lette da destra verso sinistra, a partire dal primo 1 che
compare sul nastro.
\end{defi}
\begin{defi}[Numero destro]
È l'interpretazione nei naturali del numerale destro.
\end{defi}

È facile comprendere da queste definizioni che il nastro e la testina della
macchina di Turing possono essere rappresentati da una coppia di interi, ovvero
il numero sinistro e il numero destro. Inoltre il carattere esattamente sotto
la testina è 1 se il numero destro è dispari, 0 se pari.

\begin{esempio}
Per questa macchina di Turing
\begin{align*}
\dots BB11B&111B1111BB \dots\\
&\,\uparrow
\end{align*}
il numerale sinistro è 1101, il numero sinistro è 13; il numerale destro è
1111011, il numero destro è 123.
\end{esempio}

Ora dobbiamo poter simulare su questa coppia di numeri le operazioni che la
macchina di Turing svolge sul nastro. Queste sono:
\begin{enumerate}
 \item scrivere un 1 nella casella sulla quale sta la testina;
 \item scrivere uno 0 sulla casella;
 \item spostare la testina a destra (R);
 \item spostare la testina a sinistra (L);
\end{enumerate}

Per le prime due è facile trovare una funzione ricorsiva che modifica il
numero destro in modo da rappresentare la scrittura di un 1 o di uno 0: nel
primo caso basta sommare 1 se il numero è pari (ovvero nella posizione
sottostante alla testina della MdT simulata vi è uno 0), nel secondo caso è
sufficente sottrarre 1 se il numero è dispari.

Definiamo quindi le funzioni che svolgono queste due operazioni:
$$w_1(x) = \left\{
\begin{array}{ll}
x+1 & \text{se } 2 \mid x\\
x & \text{se } 2 \nmid x
\end{array}
\right.$$
che simula l'operazione di scrivere un 1, e
$$w_B(x) = \left\{
\begin{array}{ll}
x & \text{se } 2 \mid x\\
x-1 & \text{se } 2 \nmid x
\end{array}
\right.$$
che simula l'operazione di scrivere uno 0.

Per simulare lo spostamento della testina sul nastro bisogna osservare come
cambiano i numerali quando la testina si muove a destra o a sinistra. Siano $l$
e $r$ rispettivamente il numero sinistro e il numero destro prima dello
spostamento della testina, $l'$ ed $r'$ il numero sinistro ed il numero destro
dopo lo spostamento della testina.
\begin{enumerate}
 \item Se la testina si muove verso sinistra ed $l$ è pari allora vuol
   dire che a sinistra della testina c'è uno 0, che passa da cifra
   meno significativa del numerale sinistro ad essere la cifra meno
   significativa del numerale destro.  Quindi $l' = l/2$ ed $r' = r
   \cdot 2$.
 \item Se la testina si muove verso sinistra ed $l$ è pari allora vuol
   dire che a sinistra della testina c'è un 1, che passa dal numerale
   sinistro al numerale destro; quindi $l' = (l-1)/2$ ed $r' = r \cdot
   2 + 1$.
 \item Se la testina si muove verso destra e la cifra meno
   significativa del numerale destro è un 1, questa passa al numerale
   sinistro per cui: $l' = l \cdot 2 + 1$, $r' = (r-1)/2$.
 \item Se la testina si muove verso destra e la cifra meno
   significativa del numerale destro è uno 0, $l' = l \cdot 2$ e $r' =
   r/2$.
\end{enumerate}

Vediamo ora come codificare le quintuple $\langle q_i, S_i, S_j,
\{L,R\}, q_j \rangle$ che descrivono una macchina di Turing.

Queste definiscono una funzione di transizione $\tau : \mathbb{N}^2
\rightarrow \mathbb{N}^3$ e, poichè le nostre funzioni ricorsive sono
tutte della forma $f:\mathbb{N}^k \rightarrow \mathbb{N}$, dobbiamo
trovare una funzione di codifica delle ennuple $\nu: \mathbb{N}^n
\rightarrow \mathbb{N}$ ed una classe di funzioni di proiezione
$\nu_i: \mathbb{N} \rightarrow \mathbb{N}$ che estrapolano la
proiezione dell'i-esima componente dall'ennupla codificata nei
naturali.

Delle funzioni adatte a questo scopo sono le seguenti:
$$\nu(\vec{x}) = \prod_{i=1}^n p_i^{x_i}$$ per la codifica
e $$\nu_i(w) = \max_x.p_i^x \mid w$$ per la decodifica dell'i-esima
componente, dove $p_i$ indica l'i-esimo numero primo. Queste funzioni
sfruttano l'unicità della fattorizzazione in numeri primi dei
naturali.

La funzione di codifica è chiaramente primitiva ricorsiva, in quanto
composizione di funzioni primitive ricorsive, ovvero il prodotto e
l'elevamento a potenza.

Per verificare che anche la funzione di proiezione è primitiva
ricorsiva dobbiamo introdurre gli operatori di minimizzazione e
massimizzazione limitata, rispettivamente $\mu_{w<n}[f](\vec{x})$ e
$\max_{w<n}[f](\vec{x})$, che calcolano il minimo e il massimo $w \in
[0,n]$ tale per cui $f(\vec{x}, w) = 0$.

$$\mu_{w<n}[f](\vec{x}) = \sum_{i=0}^n sgn\left(\prod_{j=0}^i
f(\vec{x},w)\right)$$

Sia $h(\vec{x},y) = f(\vec{x},n-y)$, definiamo la funzione di
massimizzazione in questo modo:

$$\max_{w<n}[f](\vec{x}) = n-\mu_{w<n}[h](\vec{x})$$

Chiaramente sia la minimizzazione che la massimizzazione limitate sono
funzioni primitive ricorsive, in quanto definite in termini di altre
funzioni primitive ricorsive.

Ora definiamo la funzione caratteristica della divisione:

\begin{align*}
div(x,y) =& \left\{
\begin{array}{ll}
1 & \text{se } x \mid y\\
0 & \text{se } x \nmid y
\end{array}
\right. \\
=& \, sgn\left( \left| \mu_{w<y+1}.\left|x \cdot w - y \right| - (y+1)\right|
\right)
\end{align*}

Infine possiamo definire la funzione di proiezione nel seguente modo:
$$\nu_i(w) = \max_{x<w}.\left| div(p_i^x, w) - 1\right|$$ e quindi
abbiamo dimostrato che anche la massimizzazione è primitiva rcorsiva.

Con gli strumenti che ci siamo appena procurati possiamo ora costruire
la codifica della funzione di transizione. Intanto ci serve una mappa
dei simboli che usiamo per descrivere la macchina di Turing nei
naturali:
\begin{align*}
q_i &\rightsquigarrow i+1\\
B &\rightsquigarrow 0\\
1 &\rightsquigarrow 1\\
L &\rightsquigarrow 1\\
R &\rightsquigarrow 2\\
\end{align*}

La funzione di transizione, dato un insieme $S$ di regole per la
macchina di Turing, è la seguente:
$$\tau(x,y) = \left\{
\begin{array}{ll}
\nu(u,v,w) & \text{se $\langle q_i, S_i, S_j, D, q_j \rangle \in
  S$}\\ & \text{e $q_i \rightsquigarrow x$, $S_i \rightsquigarrow y$,
  $S_j \rightsquigarrow u$, $D \rightsquigarrow v$, $q_j
  \rightsquigarrow w$}\\ \nu(0,0,0) & \text{se non si applica nessuna
  regola}
\end{array}
\right.$$ è una funzione definita per casi su un insieme finito di
regole $S$, per cui, come dimostrato nei paragrafi precedenti, è
primitiva ricorsiva.

Codifichiamo ora la funzione che esegue sul nastro la tripla ottenuta
dalla funzione di transizione. Siano $left(l,r)$ e $right(l,r)$ le
funzioni che simulano lo spostamento della testina rispettivamente a
sinistra ed a destra secondo le regole date in precedenza. Un passo
della MdT è codificato dalla seguente funzione:

$$step(l,r,x) = \left\{
\begin{array}{ll}
left(l, w_0(r)) \cdot p_3^{\nu_3(x)} & \text{se }\nu_1(x) = 0, \nu_2(x) = 1\\
left(l, w_1(r)) \cdot p_3^{\nu_3(x)} & \text{se }\nu_1(x) = 1, \nu_2(x) = 1\\
right(l, w_0(r)) \cdot p_3^{\nu_3(x)} & \text{se }\nu_1(x) = 0, \nu_2(x) = 2\\
right(l, w_1(r)) \cdot p_3^{\nu_3(x)} & \text{se }\nu_1(x) = 1, \nu_2(x) = 2\\
\end{array}
\right.$$

Questa funzione, definita per ricorsione primitiva, ci da lo stato
della macchina dopo $t$ passi:
\begin{align*}
f(l,r,0) &= \nu(l,r,1)\\
f(l,r,t+1) &= step(\nu_1(n), \nu_2(n), \tau(\nu_3(n), \nu_2(n) \mod 2) )
\end{align*}
con $n = f(l,r,t)$.

Vediamo come codificare una k-tupla di una funzione ricorsiva nel valore $r$.
Intanto codifichiamo un singolo elemento:
\begin{align*}
g(r,0) &= 2\cdot r + 1\\ 
g(r,x+1) &= 2\cdot f(r, x) + 1\\
\end{align*}
usando questa funzione codifichiamo la tupla:
$$cod(x_1, \dots, x_n) = g(\dots 2\cdot g(0, x_n), x_1)$$
e decodifichiamo il risultato:
$$dec(r) = \mu_x.r+1\dot{-}2^x$$

Ora non ci resta che costruire la funzione che esegue la macchina di
Turing sull'input e che ritorna la codifica del nastro alla
terminazione: $$MdT(\vec{x}) = dec(\nu_2(f(0,cod(\vec{x}),\mu_t .
|\nu_3(f(0,cod(\vec{x}),t)) = 0|)))$$

In questo modo abbiamo definito una funzione ricorsiva che svolge
l'intera computazione di una macchina di Turing, se questa termina, ed
è indefinita altrimenti. In effetti si può dire qualcosa in più: fino
alla fine della nostra costruzione abbiamo usato solo funzioni
primitive ricorsive, quindi totali per la
Proposizione~\ref{PRsonoTotali}. Ovvero è sempre possibile sapere qual
è lo stato della computazione della macchina di Turing (rappresentato
dai numerali sinistro e destro) ad un dato tempo $t$, la vera
incognita sta nella terminazione della macchina, com'era giusto
aspettarsi.

È interessante inoltre notare che la costruzione di questa funzione è
quasi del tutto indipendente dalla macchina di Turing che andiamo a
simulare con la nostra funzione ricorsiva, se non per la funzione di
transizione $\tau$ che non è altro che la rappresentazione
dell'insieme di regole date in pasto alla macchina. Potremmo dunque
benissimo pensare sostituire $\tau$ con un'altra funzione, più
sofisticata, che prende in input le quintuple codificate, lo stato
iniziale e il carattere letto e ritorna la tripla che codifica il
carattere da scrivere, la direzione in cui spostare la testina e lo
stato finale; questa funzione è in grado di prendere una qualsiasi
macchina di Turing codificata ed eseguirla su un qualsiasi
input. Inserendo questa particolare funzione di transizione e
sostituendola nel nostra prova otteniamo la macchina di Turing
universale, in grado cioè di eseguire una qualunque macchina di
Turing.

Abbiamo quindi dimostrato che per ogni macchina di Turing esiste una
funzione ricorsiva che la computa ed abbiamo visto come costruirla.
\end{proof}

\subsection{Tesi di Church-Turing}
Poichè tutti questi (ed altri) tentativi di definire tutto ciò che è
effettivamente calcolabile portano alla stessa classe di funzioni, si
può pensare che la nostra nozione intuitiva di calcolabilità coincida
esattamente con ognuna di queste definizioni.


\begin{thebibliography} {}
\bibitem{Tur} \textsc{Turing}, \textsl{On computable numbers, with an
application to the entscheidungproblem}, 1936: abelard.org.

\bibitem{Stnfrd} \textsc{Online Stanford Enciclopedia}: plato.stanford.edu.

\bibitem{CompNr} \textsl{Computable Numbers and the Turing Machine}, 1936:
turing.org.uk.

\bibitem{WTM} \textsc{Jack Copeland}, \textsl{What is a Turing Machine?}, 2000:
alanturing.net.

\bibitem{Cut} \textsc{N. Cutland}, \textsl{Computability. An
  Introduction to Recursive Function Theory}, 1980.

\bibitem{Dis} \textsc{Unipd}, \textsl{Dispense del corso di Logica 2},
  2009.
\bibitem{Wik} \textsc{Wikipedia}, \url{http://en.wikipedia.org}.
\end{thebibliography}

\end{document}
